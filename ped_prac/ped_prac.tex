%\mag1440
%\mag600
%\documentclass[draft,a4paper,12pt,reqno,oneside]{amsart}
%\documentclass[final,a4paper,12pt,reqno,oneside]{amsart, extarticle}
\documentclass[final,a4paper,14pt,reqno,oneside]{extarticle}
%\documentclass[draft,a4paper,12pt,reqno]{amsart}
%\documentclass[12pt]{article}
%\usepackage[T1]{fontenc}
\usepackage{cmap}
\usepackage[utf8]{inputenc}
\usepackage[T2A]{fontenc}
\usepackage[T2B]{fontenc}
\usepackage[T2C]{fontenc}
\usepackage[russian]{babel}
%\input glyphtounicode
%\pdfgentounicode=1
\usepackage{amsmath}
\usepackage{amssymb}
\usepackage{verbatim}
\usepackage{wasysym}
\usepackage{longtable}
\usepackage[center]{titlesec}
%\usepackage{sectsty}
%\usepackage{epic}
%\usepackage{eepic}
\usepackage{epsfig}
%\usepackage{floatflt}
\usepackage{graphicx}
%\usepackage{chapterbib}
\usepackage[nottoc]{tocbibind}
\usepackage[russian]{cleveref}

%\newcommand{\crefmiddleconjunction}{, }
%\newcommand{\creflastconjunction}{ и~}
%\newcommand{\crefrangeconjunction}{--}
%\newcommand{\crefpairconjunction}{, }
\crefname{equation}{\!\!}{\!\!}
\crefname{figure}{\!\!}{\!\!}

\linespread{1.3}

\hoffset=-10mm
\textwidth=175 mm
\textheight=263 mm
\topmargin=-20 mm
\headheight=3 mm
\headsep=10 pt
\oddsidemargin=12 mm

%\setlength{\oddsidemargin}{5 mm} \setlength{\topmargin}{0 mm}
%\setlength{\headheight}{0 mm} \setlength{\headsep}{0 mm}
%\setlength{\textwidth}{160 mm} \setlength{\textheight}{240 mm}

%\tolerance=1000
%\pagestyle{empty}

\graphicspath{{./images/}}
 


%\DeclareMathAccent{\widetilde}{\mathord}{largesymbols}{"65}
%\DeclareMathAccent{\widetilde}{\mathrel}{largesymbols}{93}
%\DeclareMathAccent{\widetilde}{\mathrel}{largesymbols}{"12}
%\DeclareMathAccent{\widetilde}{\mathord}{letters}{"5F}
%\DeclareMathAccent{\widetilde}{\mathalpha}{AMSa}{"61}
\DeclareMathAccent{\widetilde}{\mathalpha}{largesymbols}{"45}
%\DeclareMathAccent{\widehat}{\mathord}{largesymbols}{"62}
\newcommand\ff{\varphi}
\renewcommand{\f}{\varphi}
\newcommand{\ft}{\tilde\varphi}
\newcommand\eps{\varepsilon}
%\newcommand{\e}{\varepsilon}
\newcommand{\Q}{\theta}
\newcommand{\Qt}{\tilde\theta}
\newcommand{\la}{\lambda}
\newcommand{\al}{\alpha}
\newcommand{\be}{\beta}
\newcommand{\ga}{\gamma}
\newcommand{\s}{\sigma}
\newcommand{\x}{\xi}
\newcommand{\z}{\zeta}
\renewcommand{\r}{\rho}
\newcommand{\rt}{\tilde\rho}
\newcommand{\n}{\eta}
\renewcommand{\t}{\tau}
\newcommand{\er}{\bar{e}_r}
\newcommand{\F}{\mathbf{\Phi}}
\newcommand{\hU}{\mathbf{\hat{U}}}
\renewcommand{\P}{\Psi}
\newcommand{\nab}{\nabla}
\newcommand{\Lap}{\Delta}
\newcommand\om{\omega}
\newcommand\Om{\Omega}
\newcommand\Gmm{\Gamma}
\renewcommand{\k}{\varkappa}
\newcommand\dss{\displaystyle}
\newcommand\fr{\frac}
\newcommand\df{\dfrac}
\newcommand\de{\partial}
\newcommand\Op{\operatorname}
\newcommand\idot{\,\cdot}
\renewcommand{\.}{\,\cdot\,} % index dot
\newcommand{\dotm}{\!\cdot\!} % dot for scalar multiple
\newlength{\ertp}
\newcommand{\No}{№}
\renewcommand{\rt}{\tilde r}
\newcommand{\todo}{\textbf}
%\renewcommand{\sectname}{Лекция }\sectname
%\renewcommand{\cite}[1]{}
%\renewcommand{\"}{\symbol{34}}
\def\q{\quad}
\def\qq{\qquad}
%\DeclareMathOperator\div{div}
%\DeclareMathOperator\det{det}
\DeclareMathOperator{\e}{e}
\DeclareMathOperator{\diver}{div}
\DeclareMathOperator{\grad}{grad}
\DeclareMathOperator{\rot}{rot}
\DeclareMathOperator{\dif}{d}
\newcommand\diff{\dif \!}
\DeclareMathOperator{\opL}{L}
\DeclareMathOperator{\T}{T}
\DeclareMathOperator{\const}{const}
\unitlength 1.0mm \linethickness{0.4pt}


\newcommand\equationF[3]{%
\begin{equation}{\label{#1}}
 \raisebox{#3pt}{\includegraphics[scale=#2]{FIGs/#1}}\ ,
\end{equation}
}%

\newcommand{\equationFnl}[3]{%
\begin{equation*}{\label{#1}}
 \raisebox{#3pt}{\includegraphics[scale=#2]{FIGs/#1}}\ ,
\end{equation*}
}%

\newcommand\equationFF[3]{$$\raisebox{#1pt}{#3}\eqno{(#2)}$$}%

\renewcommand\thesection{\arabic{section}.}
\renewcommand\thesubsection{\thesection\arabic{subsection}.}
\renewcommand\thesubsubsection{\thesubsection\arabic{subsubsection}.}
%\allsectionsfont{\centering}
\begin{document}

\makeatletter
%\renewcommand{\thesection}{}
\renewcommand{\@oddhead}{}
\renewcommand{\@oddfoot}{\hfil \thepage \hfil}
\renewcommand{\l@section}{\@dottedtocline{1}{0em}{2.3em}} %содержание
%\renewcommand{\l@section}[2]{\hbox to\textwidth{#1\dotfill #2}}
\makeatother

%\renewcommand{\bibname}{Список литературы к лекции}
%\renewcommand\refname{Какой-то список}

\newcommand{\ssection}[1]{%
  \section[#1]{\centering\normalfont\scshape #1}}
\newcommand{\ssubsection}[1]{%
  \subsection[#1]{\raggedright\normalfont\itshape #1}}

\setlength{\parindent}{1.0cm}

\setlength{\leftmargini}{1.0cm}
\def\theenumi{\arabic{enumi}}
\def\labelenumi{\theenumi)}

%\setlength{\par1}{\parindent}
%\setlength{\parskip}{1ex}
%\parskip=2pt\parindent 0pt
\setlength{\ertp}{\parindent}



%\renewcommand{\bibname}{СПИСОК ИСПОЛЬЗОВАННЫХ ИСТОЧНИКОВ}
%\renewcommand\refname{СПИСОК ИСПОЛЬЗОВАННЫХ ИСТОЧНИКОВ}

%\input{title.tex}

%\newpage
\setcounter{page}{2}
%\thispagestyle {empty}
%\renewcommand{\contentsname}{\centering СОДЕРЖАНИЕ}
%\tableofcontents

\section*{Метод конечных разностей для решения краевых задач обыкновенных дифференциальных уравнений}
Метод конечных разностей является универсальным численным методом. Его идея заключается в том, что все производные, входящие в дифференциольное уравнение и краевые условия заменяются конечно-разностными уравнениями с использованием формул численного дифференцирования, при этом область, где требуется найти решение задачи (отрезок оси абсцисс $x$) разбивается сеткой. Дифференциальные уравнения заменяются разностными во всех внутренних узлах сетки, а краевые условия заменяются разностными только для граничных узлов. 

В результате получаем систему уравнений. Уравнений будет столько, сколько всех узлов содержит сетка. Неизвестных будет столько, сколько имеем уравнений, причем неизвестные --- решение задачи в каждом узле сетки.

Если исходная задача является линейной, то есть линейными являются и дифференциальные уравнения, и краевые условия, то приходим к системе линейных алгебраических уравнений, которую несложно решить. В случае, если определитель такой системы не равен нулю, то эта система имеет единственное решение.

Если исходная задача является нелинейной, то приходим к решению нелинейной системы уравнений. В этом случае могут возникнуть очень большие трудности, связанные с решением нелинейной системы, при этом следует иметь в виду и неединственность решения системы. 

Рассмотрим метод конечных разностей на примере решения следующей линейной задачи.
Имеем дифференциальное уравнение:
\begin{equation}\label{eq_1} 
L[u] = u''(x)+p(x)u'(x)+q(x)u(x) = f(x).
\end{equation}
Краевые условия:
\begin{equation}\label{eq_2} 
\begin{split}
\alpha_0u(a)+\alpha_1u'(a) &= A,\\
\beta_0u(b)+\beta_1u'(b) &= B.
\end{split}
\end{equation}
Здесь $p(x), q(x), f(x)$ --- известные функции, непрепывные на отрезке $[a,b]$ (где требуется найти решение задачи $u(x)$).

Видим, что задача \cref{eq_1,eq_2} линейная, так как искомая функция $u(x)$ и её производные присутствуют в уравнениях в первой степени.

Будем считать, что решение задачи \cref{eq_1,eq_2} существует и единственно. Кроме того, полагаем, что $u(x)$ есть непреывная функция и имеет непрерывные производные до 4-го порядка включительно.

Согласно методу конечных разностей, отрезок $[a, b],$ где ищем решение, разбивается сеткой
$$
a<x_0<x_1<\ldots<x_N=b,
$$
то есть сетка содержит $N+1$ узлов $x_i.$

Выберем равномерную сетку с шагом $h,$ то есть $x_i=x_0+ih, \q i=\overline{0, N},$ при этом $h=(b-a)/N.$

Исходное дифференциальное уравнение заменяем разностным уравнением для всех внутренних узлов сетки $x_i, \q i=\overline{1,N-1}.$ Для этого производные, входящие в \eqref{eq_1}, заменяем разностными отношениями по формулам численного дифференцирования. Для повышения точности будем использовать симметричные формулы:
\begin{equation*}
\begin{split}
u'(x_i) &\approx \fr{u_{i+1}-u_{i-1}}{2h},\\
u''(x_i) &\approx \fr{u_{i+1}-2u_i+u_{i-1}}{h^2}.
\end{split}
\end{equation*}
Здесь $u_i=u(x_i).$

Погрешность приведенных формул есть $\OO(h_2).$ Подставим эти формулы в уравнение \eqref{eq_1}, в результате получим разностное уравнение, вводя, как и прежде, обозначение $y_i \approx u_i.$
\begin{equation}\label{eq_3}
\fr{y_{i+1}-2y_i+y_{i-1}}{h^2} + p_i\fr{y_{i+1}-y_{i-1}}{2h}+q_iy_i = f_i, \q i=\overline{1, N-1},
\end{equation}
где $p_i = p(x_i), q_i = q(x_i), f_i = f(x_i).$

Уравнение \eqref{eq_3} справедливо только для внутренних узлов сетки и представляет собой систему $N-1$ уравнений с $N+1$ неизвестными --- $y_0, y_1, \ldots, y_N.$

Теперь хапишем разностные уравнения для краевых условий \eqref{eq_2}. Чтобы не выходить за границы отрезка $[a, b]$ воспользуемся односторонними формулами численного дифференцирования. Для граничных точек $y_0 = a, y_N = b:$
\begin{equation*}
\begin{split}
u'(a) &\approx \fr{y_1-y_0}{h},\\
u'(b) &\approx \fr{y_N-y_{N-1}}{h}.
\end{split}
\end{equation*}

Погрешность этих формул $\OO[h].$ Заметим, что если $u(x)$ есть функция достаточно гладкая, то можно использовать более точные формулы, имеющие погрешность $\OO(h^2):$
\begin{equation*}
\begin{split}
u'(a) &\approx \fr{-y_2+4y_1-3y_0}{2h},\\
u'(b) &\approx \fr{3y_N-4y_{N-1}+y_{N-2}}{h}.
\end{split}
\end{equation*}

Легко доказать справедливость последних формул. Подставляя выражения для $u'(a)$ и $u'(b)$ получаем следующие 2 уравнения:
\begin{equation}\label{eq_4}
\begin{split}
\alpha_0y_0+\alpha_1\fr{y_1-y_0}{h} &= A,\\
\beta_0y_N+\beta_1\fr{y_N-y_{N-1}}{h} &= B.
\end{split}
\end{equation}

Присоединим \eqref{eq_3} к \eqref{eq_4}. Получим полную систему линейных алгебраических уравнений. Решая полученную систему, найдем численное решение задачи $y_0, \ldots, y_N.$

\end{document}