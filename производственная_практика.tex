\input{top.tex}

\renewcommand{\bibname}{СПИСОК ИСПОЛЬЗОВАННЫХ ИСТОЧНИКОВ}
\renewcommand\refname{СПИСОК ИСПОЛЬЗОВАННЫХ ИСТОЧНИКОВ}

%\input{title.tex}

%\newpage
\setcounter{page}{2}
\thispagestyle {empty}
\renewcommand{\contentsname}{\centering СОДЕРЖАНИЕ}
\tableofcontents

\newpage
\section*{ВВЕДЕНИЕ}
\addcontentsline{toc}{section}{ВВЕДЕНИЕ}


Потребность в~изучении дифракции на~различных телах очень высока. Знания, полученные путем изучения дифракции с~помощью моделей, используются как в~гидроакустике и~эхолокации, так и~в~других областях. В~дефектоскопии основной задачей является обнаружение различных включений в~однородном теле. Это позволяет проводить исследование различных объектов методом неразрушающего контроля. Задача эхолокации~--- обнаружить и~определить местоположение объектов  по~времени задержки отражённой волны.

Для~современного мира изучение простых моделей уже не~дает требуемой точности прогнозирования поведения волн. Поэтому, необходимо изучать более сложные модели, детально описывающие рассматриваемые тела и~окружающую среду. Изучение тел с~произвольно расположенными полостями является более трудной задачей, по~сравнению с~классическими моделями. 

В~настоящей работе рассматривается задача дифракции плоских звуковых волн на~упругой сфере с произвольно расположенной полостью, заполненной жидкостью, и~радиально-неоднородным упругим покрытием. С~помощью таких покрытий можно изменять звукоотражающие свойства тел, что позволяет решать различные задачи по~формированию заданной дифракционной картины. Такой слой возможно сделать в~промышленных условиях, комбинируя несколько тонких однородных слоев с~различными механическими характеристиками. В~качестве рассматриваемого тела выбрана сфера, т.к. она является подходящей аппроксимацией для~большинства сложных тел. В~ходе работы получено аналитическое описание акустического поля, рассеяного телом.



\newpage
\section{ПОСТАНОВКА ЗАДАЧИ}
\begin{figure}[h]
\begin{center}
\begin{minipage}[h]{0.47\linewidth}
\includegraphics[width=1\linewidth]{sphere_edited.png}
\caption{<<Сфера со сферическим слоем и произвольно расположенной сферической полостью>>}\label{pic_1}
\end{minipage}
%\hfill
%\begin{minipage}[h]{0.47\linewidth}
%\includegraphics[width=1\linewidth]{2a}
%\caption{<<при $ka=3.2$>>}
%\end{minipage}
\end{center}
\end{figure}

Рассмотрим изотропный однородный упругий шар радиуса~$R\s$, плотность материала которого~$p\s$, упругие постоянные~$\la\s$ и~$\mu\s$, содержащий произвольно расположенную сферическую полость с~радиусом~$R\c$. Шар имеет покрытие в~виде неоднородного изотропного упругого слоя, внешний радиус которого равен~$R\l$ (рис. \cref{pic_1}).
 
Свяжем с полостью тела и с~самим телом прямоугольные системы координат $x\c, y\c, z\c$ и $x\s, y\s, z\s$ соответственно так, чтобы соответствующие оси обеих систем координат были параллельны. С~декартовыми системами координат $x\c, y\c, z\c$ и $x\s, y\s, z\s$ свяжем сферические координаты $r\c, \Q\c, \f\c$ и $r\s, \Q\s, \f\s$.

Пусть модули упругости~$\la\l$ и~$\mu\l$ материала слоя описываются дифференцируемыми функциями радиальной координаты~$r\s$ сферической системы координат~$(r\s, \Q\s, \f\s)$, а плотность $p\l$~--- непрерывной функцией координаты~$r\s$.  Окружающая тело и находящаяся в его полости жидкости~--- идеальные и однородные, имеющие плотности~$p\en, p\c$ и скорости звука~$c\en, c\c$ соответственно. 

Определим отраженные от~тела и возбужденные в~его~полости волны, а также найдем поля смещений в~упругом материале шара и неоднородном слое.


\newpage
\section{АНАЛИТИЧЕСКОЕ РЕШЕНИЕ ЗАДАЧИ}

Пусть из~внешнего пространства на~шар падает плоская звуковая волна. Потенциал скоростей гармонической падающей волны запишем в~виде:
\begin{equation}\label{potential speed}
\P\o(\bar{\mathbf{x}}\s, t) = A\o \exp\left[i\left(\bar{\mathbf{k}}\en\dotm\bar{\mathbf{x}}\s-\om t\right)\right],
\end{equation}
где $A\o$~--- амплитуда волны, \\
$\bar{\mathbf{k}}_e$~--- волновой вектор в~окружающей жидкости,  \\
$\lvert\bar{\mathbf{k}}\en\rvert = k\en = \om / c\en$~--- волновое число, \\
$\bar{\mathbf{x}}\s$~--- радиус-вектор, \\
$\om$~--- круговая частота.

 Без~ограничения общности будем полагать, что волна распространяется в~направлении~$\Q\s = \Q\c = 0$. Тогда в~сферической системе координат \eqref{potential speed} запишется в~виде:
\begin{equation}\label{potential speed theta = 0}
\P\o(r\s, \Q\s, t) = A\o \exp\left[i\left(k\en r\s\cos\Q\s - \om t\right)\right],
\end{equation}
В~дальнейшем временной множитель~$\exp(-i\om t)$ будем опускать.

Разложим \eqref{potential speed theta = 0} в ряд по ортогональным функциям:
\begin{equation}
\P\o(r\s, \Q\s, f\s) = \sum\limits_{n=0}^\infty\sum\limits_{m=-n}^n \gamma_{mn}j_n(k\en r\s)P_n^{\lvert m\rvert}(\cos\Q\s)\e^{im\f\s},
\end{equation}
где $j_n(x)$~--- сферическая функция Бесселя порядка $n;$\\
$P_n^m(x)$~--- присоединенный многочлен Лежандра степени $n$ порядка $m;$\\
$\gamma_{mn} = 
\begin{cases}
A\o i^n(2n+1),& \text{при } m=0;\\
0,& \text{при } m \ne 0.
\end{cases}
$

Задача определения акустических полей вне~упругого тела и внутри его~полости в~установившемся режиме колебаний заключается в~нахождении решений уравнения Гельмгольца:
\begin{align}
\Lap\P\en &+ k\en^2\P\en = 0;\label{Helmholtz_ambient}\\
\Lap\P\c &+ k\c^2\P\c = 0,\label{Helmholtz_hollow}
\end{align}
где $\P\en$~--- потенциал скоростей полного акустического поля во~внешней среде;\\
$\P\c$~--- потенциал скоростей акустического поля в~полости тела;\\
$k\c = \df{\om}{c\c}$~--- волновое число находящейся в~полости жидкости.\\ При~этом скорости частиц жидкости и акустическое давление вне~тела и внутри полости определяются по~следующим формулам соответственно:
\begin{align}
\bar{\mathbf v}\en &= \grad\P\en; &\q P\en &= ip\en\om\P\en;\label{eq_ven}\\
\bar{\mathbf v}\c &= \grad\P\c; &\q P\c &= ip\c\om\P\c.\label{eq_vc}
\end{align}


В~силу линейной постановки задачи для $\P\en$ и $\P\o$ справедливо
\begin{equation} \label{potention_speed_ambient}
\P\en = \P\o + \P\sc,
\end{equation}
где $\P\sc$~--- потенциал скоростей рассеянной звуковой волны.\\
Тогда из~\eqref{Helmholtz_ambient} получаем уравнение для нахождения~$\P\sc$:
\begin{equation} \label{Helmholtz_ambient_diffraction}
\Lap\P\sc + k\en^2\P\sc = 0.
\end{equation}

Из-за~произвольного расположения полости в теле потенциалы $\P\c$ и $\P\sc$ не~будут проявлять свойства симметрии.
Уравнения~\cref{Helmholtz_hollow,Helmholtz_ambient_diffraction} запишем в~сферических системах координат~$(r\c, \Q\c, \f\c)$ и $(r\s, \Q\s, \f\s)$ соответственно:
\begin{align}
\fr1{r\c^2}\fr\de{\de r\c}\left(r\c^2 \fr{\de\P\c}{\de r\c}\right) + \fr1{r\c^2\sin^2\Q\c}\fr{\de\P\c}{\de\f\c^2} &+ \fr1{r\c^2\sin\Q\c}\fr\de{\de\Q\c} \left(\sin\Q\c \fr{\de\P\c}{\de\Q\c}\right) + k\c^2\P\c = 0;\\
\fr1{r\s^2}\fr\de{\de r\s}\left(r\s^2 \fr{\de\P\sc}{\de r\s}\right) + \fr1{r\s^2\sin^2\Q\s}\fr{\de\P\sc}{\de\f\s^2} &+ \fr1{r\s^2\sin\Q\s}\fr\de{\de\Q\s} \left(\sin\Q\s \fr{\de\P\sc}{\de\Q\s}\right) + k\en^2\P\sc = 0.
\end{align}

Звуковая волна в~полости тела~$\P\c$ должна удовлетворять условию ограниченности, а отраженная волна~$\P\sc$~--- условиям излучения на~бесконечности. Поэтому потенциалы $\P\sc$ и $\P\c$ будем искать в~виде рядов по сферическим функциям:
\begin{align}\P\sc(r\s, \Q\s, \f\s) &= \sum\limits_{n = 0}^\infty \sum\limits_{m = 0}^n {A\sc}_{nm} h_n(k\en r\s) P_n^{\lvert m\rvert}(\cos\Q\s)\cos\bigl(m\f\s\bigr);\label{eq_psc}\\
\P\c(r\c, \Q\c, \f\c) &= \sum\limits_{n = 0}^\infty \sum\limits_{m = 0}^n {B\c}_{nm} j_n(k\c r\c) P_n^{\lvert m\rvert}(\cos\Q\c)\cos\bigl(m\f\c\bigr),\label{eq_pc}
\end{align}
где $h_n(x)$~--- сферическая функции Ханкеля первого рода.

Распространение малых возмущений в~упругом теле для~установившегося режима движения частиц тела описывается скалярным и векторным уравнением Гельмгольца:
\begin{align}
\Lap\P\s &+ {k\s}_l^2\P\s = 0;\label{Helmholtz_scalar}\\
\Lap\F\s &+ {k\s}_\t^2\F\s = 0,\label{Helmholtz_vector}
\end{align}
где ${k\s}_l$~--- волновое число продольных волн со~скоростью распространения \break 
${c\s}_l = \sqrt{\df{(\la\s + 2\mu\s)}{p\s}}$;\\
${k\s}_\t$~--- волновое число поперечных волн со~скоростью распространения \\
${c\s}_\t = \sqrt{\df{\mu\s}{p\s}}$;\\
$\P\s$ и $\F\s$~--- скалярной и векторный потенциалы смещения соответственно.

Вектор смещения $\mathbf{u}\s$ частиц упругого тела определяется по~формуле
$$
\mathbf{u}\s = \grad \P\s + \rot \F\s.
$$

Потенциал смещения $\P\s$ будем искать в~виде ряда по~двум локальным сферическим функциям:
\begin{equation}\label{eq_ps}
\begin{split}
\P\s = \sum\limits_{n=0}^\infty \sum\limits_{m=-n}^n {A\s}_{nm} h_n({k\s}_lr\c)P_n^{\lvert m\rvert}(\cos\Q\c)\e^{im\f\c} + \\
+  {B\s}_{nm} j_n({k\s}_lr\s)P_n^{\lvert m\rvert}(\cos\Q\s)\e^{im\f\s}
\end{split}
\end{equation}

Векторный потенциал $\F\s$ может быть представлен в~виде суммы:
$$
\F\s = rV\er + \rot\bigl(rW\er\bigr),
$$
где $\er$~--- орт координатной оси~$r\s$ сферической системы координат~$r\s, \Q\s, \f\s,$\\
функции $V$ и $W$ удовлетворяют скалярным уравнениям Гельмгольца
\begin{align}
\Lap V &+ {k\s}_\t^2 V = 0,\\
\Lap W &+ {k\s}_\t^2 W = 0.
\end{align}

Компоненты вектора $\mathbf u\s$ выражаются через функции $\P\s, V$ и $W$ следующим образом:
\begin{equation}\label{eq_us}
\begin{split}
{u\s}_r = \fr{\de \P\s}{\de r\s} + {k\s}_\t^2&\left(\fr{\de^2}{\de r\s^2}\bigl(r\s W\bigr) + {k\s}_\t^2r\s W\right),\\
{u\s}_\Q = \fr{1}{r\s}\fr{\de \P\s}{\de \Q\s} + \fr{{k\s}_\t^2}{r\s}&\left(\fr{{k\s}_\t r\s}{\sin \Q\s}\fr{\de V}{\de \f\s} + \fr{\de^2}{\de r\s \de \Q\s}\bigl(r\s W\bigr)\right),\\
{u\s}_\f = \fr{1}{r\s \sin\Q\s}\fr{\de \P\s}{\de \f\s} + \fr{{k\s}_\t}{r\s}&\left(\fr{1}{\sin \Q\s}\fr{\de^2 }{\de r\s \de\f\s}\bigl(r\s W\bigr) + {k\s}_\t r\s\fr{\de V}{\de \Q\s}\right).
\end{split}
\end{equation}
Функции $V$ и $W$ будем искать в~виде:
\begin{align}
V = \sum\limits_{n=0}^\infty \sum\limits_{m=-n}^n &{C\s}_{nm} h_n({k\s}_lr\c)P_n^{\lvert m\rvert}(\cos\Q\c)\e^{im\f\c} + \notag\\
+ &{D\s}_{nm} j_n({k\s}_lr\s)P_n^{\lvert m\rvert}(\cos\Q\s)\e^{im\f\s},\label{eq_v}\\
W = \sum\limits_{n=0}^\infty \sum\limits_{m=-n}^n &{E\s}_{nm} h_n({k\s}_lr\c)P_n^{\lvert m\rvert}(\cos\Q\c)\e^{im\f\c} + \notag\\
+ &{F\s}_{nm} j_n({k\s}_lr\s)P_n^{\lvert m\rvert}(\cos\Q\s)\e^{im\f\s}.\label{eq_w}
\end{align}

Распространение упругих волн в неоднородном слое описывается общими уравнениями движения упругой среды, которые для установившегося режима движения в сферической системе координат имеют следующий вид:
\begin{equation}\label{eq_moving}
\begin{split}
\fr{\de{\si\l}_{rr}}{\de r\s} &+ \fr1{r\s} \fr{\de{\si\l}_{r\Q}}{\de\Q\s} + \fr{1}{r\s\sin\Q\s}\fr{\de\si_{r\f}}{\de\f} +\\
&+ \fr1{r\s}\biggl(2{\si\l}_{rr}-{\si\l}_{\Q\Q}-{\si\l}_{\f\f}+{\si\l}_{r\Q}\ctg\Q\s\biggr)=-p\l\om^2{u\l}_r;\\
\fr{\de{\si\l}_{r\Q}}{\de r\s} &+ \fr1{r\s} \fr{\de{\si\l}_{\Q\Q}}{\de\Q\s} + \fr{1}{r\s\sin\Q\s}\fr{\de{\si\l}_{\Q\f}}{\de\f\s} +\\
&+ \fr1{r\s}\biggl(\left({\si\l}_{\Q\Q}-{\si\l}_{\f\f}\right)\ctg\Q\s+3{\si\l}_{r\Q}\biggr)\!=\!-p\l\om^2{u\l}_\Q;\\
\fr{\de{\si\l}_{r\f}}{\de r\s} &+ \fr1{r\s} \fr{\de{\si\l}_{\Q\f}}{\de\Q\s} + \fr{1}{r\s\sin\Q\s}\fr{\de{\si\l}_{\f\f}}{\de\f\s} +\\
&+ \fr1{r\s}\biggl(3{\si\l}_{r\f}+2{\si\l}_{\Q\f}\ctg\Q\s\biggr)=-p\l\om^2{u\l}_\f,
\end{split}
\end{equation}
где ${u\l}_r, {u\l}_\Q, {u\l}_\f$ --- компоненты вектора смещения $\mathbf{u\l}$;\\
${\si\l}_{ij}$ --- компоненты тензора напряжений неоднородной среды в сферической системе координат.

Используя связь компонентов тензора напряжений с компонентами тензора деформаций (обобщенный закон Гука), а также выражения компонентов тензора деформаций через компоненты вектора смещения, получаем в сферической системе координат следующие соотношения:
\begin{equation}\label{tensor comp}
    \begin{gathered}
    \begin{aligned}
        {\si\l}_{rr} &= \biggl(\la\l+2\mu\l\biggr)\fr{\de {u\l}_r}{\de r\s} +\\
        &+ \fr{\la\l} {r\s} \biggl(2{u\l}_r+\fr{\de {u\l}_\Q}{\de\Q\s}+\ctg\Q\s\;{u\l}_\Q + \fr{1}{\sin\Q\s}\fr{\de {u\l}_\f}{\de\f\s}\biggr);\\        
        {\si\l}_{\Q\Q} &= \la\l\fr{\de {u\l}_r}{\de r\s} + \fr{2(\la\l+\mu\l)}{r\s} {u\l}_r + \fr{\la\l+2\mu\l}{r\s} \fr{\de {u\l}_\Q}{\de\Q\s} +\\
         &+ \fr{\la\l}{r\s} \biggl(\ctg\Q\s \;{u\l}_\Q + \fr{1}{\sin\Q\s}\fr{\de {u\l}_\f}{\de\f\s}\biggr);\\
        {\si\l}_{\f\f} &= \la\l\fr{\de {u\l}_r}{\de r\s} + \fr{2(\la\l+\mu\l)}{r\s} {u\l}_r + \fr{\la\l}{r\s} \fr{\de {u\l}_\Q}{\de\Q\s} +\\
        &+ \fr{\la\l+2\mu\l}{r\s}\biggl(\ctg\Q\s \;{u\l}_\Q + \fr{1}{\sin\Q\s}\fr{\de {u\l}_\f}{\de\f\s}\biggr);\\
    \end{aligned}\\
    \begin{aligned}
        {\si\l}_{r\Q}=\mu\l&\left(\fr1{r\s} \fr{\de {u\l}_r}{\de\Q\s} - \fr{{u\l}_\Q}{r\s} + \fr{\de {u\l}_\Q}{\de r}\right);\\
        {\si\l}_{r\f}=\mu\l&\left(\fr1{r\s\sin\Q\s} \fr{\de {u\l}_r}{\de\f} - \fr{{u\l}_\f}{r\s} + \fr{\de {u\l}_\f}{\de r\s}\right);\\
        {\si\l}_{\Q\f}=\fr{\mu\l} {r\s}&\left(\fr1{\sin\Q\s} \fr{\de {u\l}_\Q}{\de\f\s} + \fr{\de {u\l}_\f}{\de\Q\s} - \ctg\Q\s \;{u\l}_\f \right).\\
    \end{aligned}
    \end{gathered}
\end{equation}

Соотношения \eqref{tensor comp} справедливы как для однородной упругой среды, так и для неоднородного слоя. В первом случае в выражениях \eqref{tensor comp} компоненты вектора смещения и тензора напряжений, а также величины $\la\l$ и $\mu\l$ следует заменить на $\la\s$ и $\mu\s$ соответственно.

Используя эти соотношения, запишем уравнения~\eqref{eq_moving} через компоненты вектора смещения $\mathbf{u\l}$:
\begin{equation*}
\begin{split}
\biggl(\la\l+2\mu\l\biggr)\fr{\de^2 {u\l}_r}{\de r\s^2} + \fr{\mu\l}{r\s^2}\fr{\de^2 {u\l}_r}{\de\Q\s^2} +\fr{\mu\l}{r\s\sin\Q\s}\fr{\de^2 {u\l}_r}{\de\f\s^2} + \fr{\mu\l\ctg\Q\s}{r\s^2}\fr{\de {u\l}_r}{\de \Q\s} + \\
+ \left(\la\l'+2\mu\l'+\fr{2\la\l+4\mu\l}{r\s}\right)\fr{\de{u\l}_r}{\de r\s} + \left(\fr{2\la\l'}{r\s} - \fr{2\la\l+4\mu\l}{r\s^2}\right){u\l}_r + \\
+ \fr{\la\l+\mu\l}{r\s}\fr{\de^2 {u\l}_\Q}{\de r\s\de\Q\s} + \fr{(\la\l+\mu\l)\ctg\Q\s}{r\s}\fr{\de {u\l}_\Q}{\de r\s} + \\ 
+ \left(\fr{\la\l'}{r\s} - \fr{\la\l+3\mu\l}{r\s^2}\right)\fr{\de {u\l}_\Q}{\de \Q\s} +\left(\fr{\la\l'\ctg\Q\s}{r\s} - \fr{(\la\l+3\mu\l)\ctg\Q\s}{r\s^2}\right){u\l}_\Q + \\
+\fr{\la\l+\mu\l}{r\s\sin\Q\s}\fr{\de^2 {u\l}_\f}{\de r\s\de\f\s}
+ \left(\fr{\la\l'}{r\s\sin\Q\s} - \fr{\la\s+3\mu\s}{r\s^2\sin\Q\s}\right)\fr{\de {u\l}_\f}{\de \f\s} = -p\l\om^2{u\l}_r;
\end{split}
\end{equation*}

\begin{equation*}
\begin{split}
\fr{\la\l+\mu\l}{r\s}\fr{\de^2 {u\l}_r}{\de r\s \de \Q\s} + \left(\fr{\mu\l'}{r\s}+\fr{2\la\l+4\mu\l}{r\s^2}\right)\fr{\de {u\l}_r}{\de \Q\s} + \\
+ \mu\l \fr{\de^2 {u\l}_\Q}{\de r\s^2} + \fr{\la\l+2\mu\l}{r\s^2}\fr{\de^2 {u\l}_\Q}{\de \Q\s^2} + \fr{\mu\l}{r\s^2\sin^2\Q\s} \fr{\de^2 {u\l}_\Q}{\de \f^2} + \\
+ \left(\mu\l'+ \fr{2\mu\l}{r\s}\right)\fr{\de {u\l}_\Q}{\de r\s} + \fr{(\la\l+2\mu\l)\ctg\Q\s}{r\s^2}\fr{\de{u\l}_\Q}{\de \Q\s}-\left(\fr{\mu\l'}{r\s} + \fr{\la\l+2\mu\l}{r\s^2\sin^2\Q\s}\right){u\l}_\Q + \\
+ \fr{\la\l+\mu\l}{r\s^2\sin\Q\s}\fr{\de^2{u\l}_\f}{\de \Q\s \de \f\s} - \fr{(\la\l+3\mu\l)\ctg\Q\s}{r\s^2\sin\Q\s}\fr{\de {u\l}_\f}{\de \f\s}  = -p\l\om^2{u\l}_\Q;
\end{split}
\end{equation*}

\begin{equation*}
\begin{split}
\fr{\la\l+\mu\l}{r\s\sin\Q\s} \fr{\de^2{u\l}_r}{\de r\s \de \f\s} + \left(\fr{\mu\l'}{r\s\sin\Q\s} + \fr{2\la\l+4\mu\l}{r\s^2\sin\Q\s} \right)\fr{\de {u\l}_r}{\de \f\s} +\\
+ \fr{\la\l+\mu\l}{r\s^2\sin\Q\s}\fr{\de^2 {u\l}_\Q}{\de \Q\s \de \f\s} + \fr{(\la\l+3\mu\l)\ctg\Q\s}{r\s^2\sin\Q\s}\fr{\de {u\l}_\Q}{\de \f\s} +\\
+ \mu\l \fr{\de^2 {u\l}_\f}{\de r\s^2} + \fr{\mu\l}{r\s^2}\fr{\de^2 {u\l}_\f}{\de \Q\s^2} + \fr{\la\l+2\mu\l}{r\s^2\sin^2\Q\s} \fr{\de^2 {u\l}_\f}{\de \f\s^2} + \\
+ \left(\mu\l'+\fr{2\mu\l}{r\s}\right)\fr{\de {u\l}_\f}{\de r\s} + \fr{\mu\l\ctg\Q\s}{r\s^2}\fr{\de {u\l}_\f}{\de \Q\s} - \left(\fr{\mu\l'}{r\s} + \fr{\mu\l}{r\s^2\sin^2\Q\s}\right){u\l}_\f = - p\l\om^2{u\l}_\f.
\end{split}
\end{equation*}
Введем новые функции ${u\l}_2$ и ${u\l}_3$, связанные с ${u\l}_\Q$ и ${u\l}_\f$ следующими соотношениями:
\begin{equation}\label{eq_v2v3}
{u\l}_\Q = \df{\de {u\l}_2}{\de\Q\s} + \fr1{\sin\Q\s}\fr{\de {u\l}_3}{\de\f\s}; \qq {u\l}_\f = \df1{\sin\Q\s}\df{\de {u\l}_2}{\de\f\s} - \df{\de {u\l}_3}{\de\Q\s},
\end{equation}
запишем уравнения движения через функции ${u\l}_r, {u\l}_2$ и ${u\l}_3$ и перенесем всё в левую часть:

\begin{equation}\label{eq_1.1}
\begin{split}
\biggl(\la\l+2\mu\l\biggr)\fr{\de^2 {u\l}_r}{\de r\s^2} + \left(\la\l'+2\mu\l'+\fr{2\la\l+4\mu\l}{r\s}\right)\fr{\de {u\l}_r}{\de r\s} + \fr{\mu\l}{r\s^2}L({u\l}_r) +\\
+ \left(\fr{2\la\l'}{r\s} - \fr{2\la\l+4\mu\l}{r\s^2} + p\l\om^2\right){u\l}_r +\\
+ \left[\fr{\la\l+\mu\l}{r\s}\fr{\de}{\de r\s} + \fr{\la\l'}{r\s} - \fr{\la\l+3\mu\l}{r\s}\right]L({u\l}_3) = 0;
\end{split}
\end{equation}
\begin{equation}\label{eq_2.1}
\begin{split}
\left[\fr{\la\l+\mu\l}{r\s}\fr{\de}{\de r\s}+\fr{\mu\l'}{r\s}+\fr{2\la\l+4\mu\l}{r\s^2}\right]\fr{\de{u\l}_r}{\de \Q\s} + \fr{\la\l+2\mu\l}{r\s^2}\fr{\de}{\de \Q\s}L({u\l}_2) + \\
+ \left[\mu\l\fr{\de^2}{\de r\s^2} + \left(\mu\l'+\fr{2\mu\l}{r\s}\right)\fr{\de}{\de r\s} - \fr{\mu\l'}{r\s} + p\l\om^2\right]\left[\fr{\de {u\l}_2}{\de\Q\s} + \fr{1}{\sin\Q\s}\fr{\de{u\l}_3}{\de\f\s}\right]+\\
+ \fr{\mu\l}{r\s^2\sin\Q\s}\fr{\de}{\de\f\s}L({u\l}_3) = 0;
\end{split}
\end{equation}
\begin{equation}\label{eq_3.1}
\begin{split}
\fr{1}{r\s\sin\Q\s}\left[\biggl(\la\l+\mu\l\biggr)\fr{\de}{\de r\s} + \mu\l' + \fr{2\la\l+4\mu\l}{r\s}\right]\fr{\de {u\l}_r}{\de\f\s} +\\
+ \left[\mu\l\fr{\de^2}{\de r\s^2} + \left(\mu\l'+\fr{2\mu\l}{r\s}\right)\fr{\de}{\de r\s} - \fr{\mu\l'}{r\s} + p\l\om^2\right]\left[\fr{1}{\sin\Q\s}\fr{\de {u\l}_2}{\de\f\s} - \fr{\de {u\l}_3}{\de\Q\s}\right] +\\
+ \fr{\la\l+2\mu\l}{r\s^2\sin\Q\s}\fr{\de}{\de \f\s}L({u\l}_2) -\fr{\mu\l}{r\s^2}\fr{\de}{\de\Q\s}L({u\l}_3) = 0,
\end{split}
\end{equation}
где $L = \df{\de^2}{\de\Q\s^2} + \ctg\Q\s\df{\de}{\de\Q\s} + \df{1}{\sin^2\Q\s}\df{\de^2}{\de\f\s^2},\q\la\l' = \df{\diff\la\l}{\diff r\s},\q\mu\l' = \df{\diff\mu\l}{\diff r\s}.$

Сделаем преобразования уравнений. Уравнение \cref{eq_2.1} домножим на $\sin(\Q\s)$ и продифференцируем по $\Q\s$, а уравнение \cref{eq_3.1} продифференцируем по $\f\s.$ Сложим полученные уравнения и разделим на $\sin(\Q\s)$. Получим следующее уравнение:
\begin{equation}\label{eq_2.2}
\begin{split}
\left[\fr{\la\l+\mu\l}{r\s}\fr{\de}{\de r\s} + \fr{\mu\l'}{r\s} + \fr{2\la\l+4\mu\l}{r\s^2}\right]&L({u\l}_r)+\\
\left[\mu\l\fr{\de^2}{\de r\s^2} + \left(\mu\l'+\fr{2\mu\l}{r\s}\right)\fr{\de}{\de r\s} - \fr{\mu\l'}{r\s} + p\l\om^2\right]&L({u\l}_2) + \\
 + \fr{\la\l+2\mu\l}{r\s^2}&L^2({u\l}_2) = 0,
\end{split}
\end{equation}
не содержащее функцию ${u\l}_3.$ Затем продифференцируем уравнение \cref{eq_2.1} по $\f\s$ и вычтем уравнение \cref{eq_3.1}, предварительно умноженное на $\sin(\Q\s)$ и продифференцированное по $\Q\s,$ а затем всё разделим на $\sin(\Q\s).$ Получим следующее уравнение:
\begin{equation}\label{eq_3.2}
\begin{split}
\left[\mu\fr{\de^2}{\de r\s^2} + \left(\mu\l' + \fr{2\mu\l}{r\s}\right)\fr{\de}{\de r\s} - \fr{\mu\l'}{r\s} + p\l\om^2\right]&L({u\l}_3) + \\
\fr{\mu\l}{r\s^2}&L^2({u\l}_3) = 0,
\end{split}
\end{equation}
в котором отсутствуют функции ${u\l}_r$ и ${u\l}_2.$

В результате получили систему, состоящую из уравнений \cref{eq_1.1,eq_2.2,eq_3.2}.

Функции ${u\l}_r, {u\l}_2$ и ${u\l}_3$ также будем искать в виде разложений по сферическим функциям:
\begin{equation}\label{eq_uru2u3}
\begin{split}
{u\l}_r(r\s, \Q\s, \f\s) = \sum\limits_{n=0}^\infty\sum\limits_{m=-n}^n U_{1mn}(r\s)P_n^{\lvert m\rvert}(\cos\Q\s)\e^{im\f\s},\\
{u\l}_2(r\s, \Q\s, \f\s) = \sum\limits_{n=0}^\infty\sum\limits_{m=-n}^n U_{2mn}(r\s)P_n^{\lvert m\rvert}(\cos\Q\s)\e^{im\f\s},\\
{u\l}_3(r\s, \Q\s, \f\s) = \sum\limits_{n=0}^\infty\sum\limits_{m=-n}^n U_{3mn}(r\s)P_n^{\lvert m\rvert}(\cos\Q\s)\e^{im\f\s}.\\
\end{split}
\end{equation}

Коэффициенты ${A\sc}_{nm}, {B\c}_{nm}, {A\s}_{nm}, {B\s}_{nm}, {C\s}_{nm}, {D\s}_{nm}, {E\s}_{nm}$ и $ {F\s}_{nm}$ разложений \cref{eq_psc,eq_ps,eq_pc,eq_v,eq_w}, а также функции $U_{1mn},U_{2mn}$ и $U_{3mn}$ из разложений \cref{eq_uru2u3} подлежат определению из граничных условий, которые заключаются в равенстве нормальных скоростей частиц упругой среды и жидкости на внешней поверхности слоя и внутренней поверхности полого шара; равенстве на них нормального напряжения и акустического давления; отсутствии на этих поверхностях касательных напряжений. На внутренней поверхности слоя при переходе через границу раздела упругих сред должны быть непрерывны составляющие вектора смещения частиц, а также нормальные и тангенциальные напряжения. Имеем:
\begin{align}
\text{при }r\c &= R\c: \q  &  {\si\s}_{rr} &= -P\c,  &  {\si\s}_{r\Q} &= 0,  &  {\si\s}_{r\f} &= 0, \notag\\
&&  -i\om {u\s}_r &= {v\c}_r;\label{eq_border_1.1}\\
\text{при }r\s &= R\s: \q  &  {\si\s}_{rr} &= {\si\l}_{rr},  &  {\si\s}_{r\Q} &= {\si\l}_{r\Q},  &  {\si\s}_{r\f} &= {\si\l}_{r\f}, &&\notag\\
&&  {u\s}_r &= {u\l}_r, \q & {u\s}_\Q &= {u\l}_\Q, \q & {u\s}_\f &= {u\l}_\f;\label{eq_border_2.1}\\
\text{при }r\s &= R\l: \q  &  {\si\l}_{rr} &= -P\en,  &  {\si\l}_{r\Q} &= 0,  &  {\si\l}_{r\f} &= 0, \notag\\
&&  -i\om {u\l}_r &= {v\en}_{r},\label{eq_border_3.1}
\end{align} 
где ${v\c}_r=\de\P\c/\de r, \q {v\en}_r=\de\P\en/\de r$~--- радиальная компонента скорости частиц в жидкости внутри полости $\bar{\mathbf {v}}\c$ и в окружающем пространстве $\bar {\mathbf v}\en$ соответственно.

Используя выражения \cref{eq_ven,tensor comp,eq_v2v3}, запишем граничные условия \cref{eq_border_3.1} через функции $\P\o, \P\sc, {u\l}_r, {u\l}_2$ и ${u\l}_3.$ Получим для $r\s = R\l:$
\begin{align}
-i\om{u\l}_r &= \fr{\de(\P\o+\P\sc)}{\de r\s} ;\label{eq_Rl.1}\\
\biggl(\la\l+2\mu\l\biggr)\fr{\de {u\l}_r}{\de r\s} + \fr{2\la\l}{r\s}{u\l}_r + \fr{\la\l}{r\s}L({u\l}_r) &= ip\en\om\left(\P\o+\P\sc\right);\label{eq_Rl.2}\\
\fr{\mu\l}{r\s}\fr{\de ({u\l}_r - {u\l}_2)}{\de \Q\s} + \mu\l\fr{\de^2 {u\l}_2}{\de r\s\de \Q\s} + \fr{\mu\l}{\sin\Q\s}\fr{\de}{\de \f\s}&\left(\fr{\de {u\l}_3}{\de r\s} - \fr{{u\l}_3}{r\s}\right) = 0;\label{eq_Rl.3}\\
\fr{\mu\l}{r\s\sin\Q\s}\fr{\de}{\de \f\s}\biggl({u\l}_r-{u\l}_2\biggr) + \fr{\mu\l}{\sin\Q\s}\fr{\de^2 {u\l}_2}{\de r\s \de \f\s} - \mu\l\fr{\de}{\de \Q\s}&\left(\fr{\de {u\l}_3}{\de r\s} - \fr{{u\l}_3}{r}\right) = 0.\label{eq_Rl.4}
\end{align}

Аналогично, используя выражения \cref{eq_us,tensor comp,eq_v2v3}, запишем граничные условия \cref{eq_border_2.1} через функции ${u\s}_r, {u\s}_2, {u\s}_3, \P\s, V$ и $W,$ а граничные условия \cref{eq_border_1.1} через $\P\s, V$ и $W.$

Подставив разложения \cref{eq_uru2u3} в уравнения \cref{eq_1.1,eq_2.2,eq_3.2}, воспользовавшись уравнением для присоединенных многочленов Лежандра и свойством ортогональности этих многочленов, получим для каждой пары индексов $m,n \q(n = 0,1,\ldots; \lvert m\rvert\leq n)$ систему линейных однородных обыкновенных дифференциальных уравнений второго порядка относительно неизвестных функций \\ $U_{1mn}(r\s), U_{mn}(r\s)$ и $U_{3mn}(r\s).$ А подставив разложения \cref{eq_uru2u3} в полученные граничные условия, найдем краевые условия и сможем решить  систему дифференциальных уравнений. После ее решения определим коэффициенты \\
${A\sc}_{nm}, {B\c}_{nm}, {A\s}_{nm}, {B\s}_{nm}, {C\s}_{nm}, {D\s}_{nm}, {E\s}_{nm}$ и $ {F\s}_{nm}$ разложений  \cref{eq_psc,eq_ps,eq_pc,eq_v,eq_w} для каждой пары индексов $m,n,$ а зная коэффициенты ${A\sc}_{nm}$, по формуле \cref{eq_psc} найдем акустическое поле, рассеяное упругой сферой, имеющей произвольно расположенную полость и неоднородное покрытие.



\newpage
\section*{ЗАКЛЮЧЕНИЕ}
\addcontentsline{toc}{section}{ЗАКЛЮЧЕНИЕ}
Была поставлена задача о дифракции плоских звуковых волн на упругой сфере, имеющей произвольно расположенную полость и неоднородное покрытие. В данной работе приведены основные уравнения колебаний, разложения в ряд искомых функций, а также способ получения системы линейных однородных обыкновенных дифференциальных уравнений второго порядка и краевых условий.



\end{document}
