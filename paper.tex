%\mag1440
%\mag600
%\documentclass[draft,a4paper,12pt,reqno,oneside]{amsart}
%\documentclass[final,a4paper,12pt,reqno,oneside]{amsart, extarticle}
\documentclass[final,a4paper,14pt,reqno,oneside]{extarticle}
%\documentclass[draft,a4paper,12pt,reqno]{amsart}
%\documentclass[12pt]{article}
%\usepackage[T1]{fontenc}
\usepackage{cmap}
\usepackage[utf8]{inputenc}
\usepackage[T2A]{fontenc}
\usepackage[T2B]{fontenc}
\usepackage[T2C]{fontenc}
\usepackage[russian]{babel}
%\input glyphtounicode
%\pdfgentounicode=1
\usepackage{amsmath}
\usepackage{amssymb}
\usepackage{verbatim}
\usepackage{wasysym}
\usepackage{longtable}
\usepackage[center]{titlesec}
%\usepackage{sectsty}
%\usepackage{epic}
%\usepackage{eepic}
\usepackage{epsfig}
%\usepackage{floatflt}
\usepackage{graphicx}
%\usepackage{chapterbib}
\usepackage[nottoc]{tocbibind}
\usepackage[russian]{cleveref}

%\newcommand{\crefmiddleconjunction}{, }
%\newcommand{\creflastconjunction}{ и~}
%\newcommand{\crefrangeconjunction}{--}
%\newcommand{\crefpairconjunction}{, }
\crefname{equation}{\!\!}{\!\!}
\crefname{figure}{\!\!}{\!\!}

\linespread{1.3}

\hoffset=-10mm
\textwidth=175 mm
\textheight=263 mm
\topmargin=-20 mm
\headheight=3 mm
\headsep=10 pt
\oddsidemargin=12 mm

%\setlength{\oddsidemargin}{5 mm} \setlength{\topmargin}{0 mm}
%\setlength{\headheight}{0 mm} \setlength{\headsep}{0 mm}
%\setlength{\textwidth}{160 mm} \setlength{\textheight}{240 mm}

%\tolerance=1000
%\pagestyle{empty}

\graphicspath{{./images/}}
 


%\DeclareMathAccent{\widetilde}{\mathord}{largesymbols}{"65}
%\DeclareMathAccent{\widetilde}{\mathrel}{largesymbols}{93}
%\DeclareMathAccent{\widetilde}{\mathrel}{largesymbols}{"12}
%\DeclareMathAccent{\widetilde}{\mathord}{letters}{"5F}
%\DeclareMathAccent{\widetilde}{\mathalpha}{AMSa}{"61}
\DeclareMathAccent{\widetilde}{\mathalpha}{largesymbols}{"45}
%\DeclareMathAccent{\widehat}{\mathord}{largesymbols}{"62}
\newcommand\ff{\varphi}
\renewcommand{\f}{\varphi}
\newcommand{\ft}{\tilde\varphi}
\newcommand\eps{\varepsilon}
%\newcommand{\e}{\varepsilon}
\newcommand{\Q}{\theta}
\newcommand{\Qt}{\tilde\theta}
\newcommand{\la}{\lambda}
\newcommand{\al}{\alpha}
\newcommand{\be}{\beta}
\newcommand{\ga}{\gamma}
\newcommand{\s}{\sigma}
\newcommand{\x}{\xi}
\newcommand{\z}{\zeta}
\renewcommand{\r}{\rho}
\newcommand{\rt}{\tilde\rho}
\newcommand{\n}{\eta}
\renewcommand{\t}{\tau}
\newcommand{\er}{\bar{e}_r}
\newcommand{\F}{\mathbf{\Phi}}
\newcommand{\hU}{\mathbf{\hat{U}}}
\renewcommand{\P}{\Psi}
\newcommand{\nab}{\nabla}
\newcommand{\Lap}{\Delta}
\newcommand\om{\omega}
\newcommand\Om{\Omega}
\newcommand\Gmm{\Gamma}
\renewcommand{\k}{\varkappa}
\newcommand\dss{\displaystyle}
\newcommand\fr{\frac}
\newcommand\df{\dfrac}
\newcommand\de{\partial}
\newcommand\Op{\operatorname}
\newcommand\idot{\,\cdot}
\renewcommand{\.}{\,\cdot\,} % index dot
\newcommand{\dotm}{\!\cdot\!} % dot for scalar multiple
\newlength{\ertp}
\newcommand{\No}{№}
\renewcommand{\rt}{\tilde r}
\newcommand{\todo}{\textbf}
%\renewcommand{\sectname}{Лекция }\sectname
%\renewcommand{\cite}[1]{}
%\renewcommand{\"}{\symbol{34}}
\def\q{\quad}
\def\qq{\qquad}
%\DeclareMathOperator\div{div}
%\DeclareMathOperator\det{det}
\DeclareMathOperator{\e}{e}
\DeclareMathOperator{\diver}{div}
\DeclareMathOperator{\grad}{grad}
\DeclareMathOperator{\rot}{rot}
\DeclareMathOperator{\dif}{d}
\newcommand\diff{\dif \!}
\DeclareMathOperator{\opL}{L}
\DeclareMathOperator{\T}{T}
\DeclareMathOperator{\const}{const}
\unitlength 1.0mm \linethickness{0.4pt}


\newcommand\equationF[3]{%
\begin{equation}{\label{#1}}
 \raisebox{#3pt}{\includegraphics[scale=#2]{FIGs/#1}}\ ,
\end{equation}
}%

\newcommand{\equationFnl}[3]{%
\begin{equation*}{\label{#1}}
 \raisebox{#3pt}{\includegraphics[scale=#2]{FIGs/#1}}\ ,
\end{equation*}
}%

\newcommand\equationFF[3]{$$\raisebox{#1pt}{#3}\eqno{(#2)}$$}%

\renewcommand\thesection{\arabic{section}.}
\renewcommand\thesubsection{\thesection\arabic{subsection}.}
\renewcommand\thesubsubsection{\thesubsection\arabic{subsubsection}.}
%\allsectionsfont{\centering}
\begin{document}

\makeatletter
%\renewcommand{\thesection}{}
\renewcommand{\@oddhead}{}
\renewcommand{\@oddfoot}{\hfil \thepage \hfil}
\renewcommand{\l@section}{\@dottedtocline{1}{0em}{2.3em}} %содержание
%\renewcommand{\l@section}[2]{\hbox to\textwidth{#1\dotfill #2}}
\makeatother

%\renewcommand{\bibname}{Список литературы к лекции}
%\renewcommand\refname{Какой-то список}

\newcommand{\ssection}[1]{%
  \section[#1]{\centering\normalfont\scshape #1}}
\newcommand{\ssubsection}[1]{%
  \subsection[#1]{\raggedright\normalfont\itshape #1}}

\setlength{\parindent}{1.0cm}

\setlength{\leftmargini}{1.0cm}
\def\theenumi{\arabic{enumi}}
\def\labelenumi{\theenumi)}

%\setlength{\par1}{\parindent}
%\setlength{\parskip}{1ex}
%\parskip=2pt\parindent 0pt
\setlength{\ertp}{\parindent}



\renewcommand{\bibname}{СПИСОК ИСПОЛЬЗОВАННЫХ ИСТОЧНИКОВ}
\renewcommand\refname{СПИСОК ИСПОЛЬЗОВАННЫХ ИСТОЧНИКОВ}

%\input{title.tex}

%\newpage
\setcounter{page}{5}
\thispagestyle {empty}
\renewcommand{\contentsname}{\centering СОДЕРЖАНИЕ}
\tableofcontents

\newpage
\section*{ВВЕДЕНИЕ}
\addcontentsline{toc}{section}{ВВЕДЕНИЕ}
Потребность в~изучении дифракции на~различных телах очень высока. Знания, полученные путем изучения дифракции с~помощью моделей, используются как в~гидроакустике и эхолокации, так и в~других областях. В~дефектоскопии основной задачей является обнаружение различных включений в~однородном теле. Это позволяет проводить исследование различных объектов методом неразрушающего контроля.

В~настоящей работе рассматривается задача дифракции плоских звуковых волн на~упругой сфере, имеющей произвольно расположенную полость и неоднородное покрытие. С~помощью таких покрытий можно изменять звукоотражающие свойства тел. В~качестве рассматриваемого тела выбрана сфера, т.к. более сложные тела можно аппроксиммировать с~помощью сферы. 


\newpage

\section{Постановка задачи.} Рассмотрим изотропный однородный упругий шар радиуса~$r_2$, плотность материала которого~$\r_2$, упругие постоянные~$\la_2$ и~$\mu_2$, содержащий произвольно расположенную сферическую полость с~радиусом~$r_1$. Шар имеет покрытие в~виде неоднородного изотропного упругого слоя, внешний радиус которого равен~$r_3$. Полагаем, что модули упругости~$\la_3$ и~$\mu_3$ материала слоя описываются дифференцируемыми функциями радиальной координаты~$r$ сферической системы координат~$(r, \Q, \f)$, а плотность $\r_3$~--- непрерывной функцией координаты~$r$.  Окружающая тело и находящаяся в его полости жидкости~--- идеальные и однородные, имеющие плотности~$\r_0, \r_1$ и скорости звука~$c_0, c_1$ соответственно. 

Свяжем с полостью тела и с~самим телом прямоугольные системы координат $x_1, y_1, z_1$ и $x_2, y_2, z_2$ соответственно так, чтобы соответствующие оси обеих систем координат были параллельны. С~декартовыми системами координат $x_1, y_1, z_1$ и $x_2, y_2, z_2$ свяжем сферические координаты $r_1, \Q_1, \f_1$ и $r_2, \Q_2, \f_2$.

Пусть из~внешнего пространства на~шар падает плоская звуковая волна. Потенциал скоростей гармонической падающей волны запишем в~виде:
\begin{equation}\label{potential speed}
\P_0(\bar{\mathbf{x}}, t) = A_0 \exp\left[i\left(\bar{\mathbf{k}}\dotm\bar{\mathbf{x}}-\om t\right)\right],
\end{equation}
где $A_0$~--- амплитуда волны, $\bar{\mathbf{k}}$~--- волновой вектор в~окружающей жидкости,  $\lvert\bar{\mathbf{k}}\rvert = k = \om / c$~--- волновое число, $\bar{\mathbf{x}}$~--- радиус-вектор, $\om$~--- круговая частота.

 Без~ограничения общности будем полагать, что волна распространяется в~направлении~$\Q_1 = \Q_2 = 0$. Тогда в~сферической системе координат \eqref{potential speed} запишется в~виде:
\begin{equation}
\P_0(r_2, \Q_2, t) = A_0 \exp\left[i\left(kr_2\cos\Q_2 - \om t\right)\right],
\end{equation}
В~дальнейшем временной множитель~$\exp(-i\om t)$ будем опускать.

Определим отраженные от~тела и возбужденные в~его~полости волны, а также найдем поля смещений в~упругом материале шара и неоднородном слое.

\section{Определение волновых полей}
Задача определения акустических полей вне~упругого тела и внутри его~полости в~установившемся режиме колебаний заключается в~нахождении решений уравнения Гельмгольца:
\begin{align}
\Lap\P^{(0)} &+ k_0^2\P^{(0)} = 0;\label{Helmholtz_ambient}\\
\Lap\P^{(1)} &+ k_1^2\P^{(1)} = 0,\label{Helmholtz_hollow}
\end{align}
где $\P^{(0)}$~--- потенциал скоростей полного акустического поля во~внешней среде;\\
$\P^{(1)}$~--- потенциал скоростей акустического поля в~полости тела;\\
$k_0 = \df{\om}{c_0}$ и $k_1 = \df{\om}{c_1}$~--- волновые числа окружающей тело и находящейся в~полости жидкостей соответственно. При~этом скорости частиц жидкости и акустическое давление вне~тела ($j = 0$) и внутри полости ($j = 1$) определяются по~следующим формулам соответственно:
\begin{equation}
\bar{v}_j = \grad\P^{(j)}; \q p_j = i\r_j\om\P^{(j)} \qq (j = 0, 1).
\end{equation}

Потенциал скоростей падающей плоской волны представим в~виде
\begin{equation}
\P_0(r, \Q) = A_0 \sum\limits_{n = 0}^{\infty} (2n + 1) i^n j_n(kr) P_n(\cos\Q),
\end{equation}
где $j_n(x)$~--- сферическая функция Бесселя порядка $n$, \\
$P_n(x)$~--- многочлен Лежандра степени $n$.


В~силу линейной постановки задачи $\P^{(1)}$ и $\P_s$
\begin{equation} \label{potention_speed_ambient}
\P^{(0)} = \P_0 + \P_s,
\end{equation}
где $\P_s$~--- потенциал скоростей рассеянной звуковой волны. Тогда из~\eqref{Helmholtz_ambient} получаем уравнение для нахождения~$\P_s$:
\begin{equation} \label{Helmholtz_ambient_diffraction}
\Lap\P_s + k_0^2\P_s = 0.
\end{equation}

Из-за~произвольного расположения полости в теле потенциалы $\P^{(1)}$ и $\P_s$ не~будут проявлять свойства симметрии.
Уравнения~\cref{Helmholtz_hollow,Helmholtz_ambient_diffraction} запишем в~сферических системах координат~$(r_1, \Q_1, \f_1)$ и $(r_2, \Q_2, \f_2)$ соответственно:
\begin{align}
\fr1{r_1^2}\fr\de{\de r_1}\left(r_1^2 \fr{\de\P^{(1)}}{\de r_1}\right) + \fr1{r_1^2\sin^2\Q_1}\fr{\de\P^{(1)}}{\de\f_1^2} &+ \fr1{r_1^2\sin\Q_1}\fr\de{\de\Q_1} \left(\sin\Q_1 \fr{\de\P^{(1)}}{\de\Q_1}\right) + k_1^2\P^{(1)} = 0;\\
\fr1{r_2^2}\fr\de{\de r_2}\left(r_2^2 \fr{\de\P_s}{\de r_2}\right) + \fr1{r_2^2\sin^2\Q_2}\fr{\de\P_s}{\de\f_2^2} &+ \fr1{r_2^2\sin\Q_2}\fr\de{\de\Q_2} \left(\sin\Q_2 \fr{\de\P_s}{\de\Q_2}\right) + k_0^2\P_s = 0;
\end{align}
Звуковая волна в~полости тела~$\P^{(1)}$ должна удовлетворять условию ограниченности, а отраженная волна~$\P_s$~--- условиям излучения на~бесконечности. Поэтому потенциалы $\P^{(1)}$ и $\P_s$ будем искать в~виде 
\begin{align}
\P^{(1)} = \sum\limits_{n = 0}^\infty \sum\limits_{m = -n}^n A_{nm} j_n(k_1 r_1) P_n^m(\cos\Q_1)\e^{im\f_1};\\
\P_s = \sum\limits_{n = 0}^\infty \sum\limits_{m = -n}^n B_{nm} h_n(k_0 r_2) P_n^m(\cos\Q_2)\e^{im\f_2},
\end{align}
где $h_n(x)$~--- сферическая функция Ганкеля первого рода порядка $n$.

\end{document}