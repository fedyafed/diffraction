\input{top.tex}

\renewcommand{\bibname}{СПИСОК ИСПОЛЬЗОВАННЫХ ИСТОЧНИКОВ}
\renewcommand\refname{СПИСОК ИСПОЛЬЗОВАННЫХ ИСТОЧНИКОВ}

%\input{title.tex}

%\newpage
\setcounter{page}{2}
\thispagestyle {empty}
\renewcommand{\contentsname}{\centering СОДЕРЖАНИЕ}
\tableofcontents

\newpage
\section*{ВВЕДЕНИЕ}
\addcontentsline{toc}{section}{ВВЕДЕНИЕ}
Потребность в~изучении дифракции на~различных телах очень высока. Знания, полученные путем изучения дифракции с~помощью моделей, используются как в~гидроакустике и эхолокации, так и в~других областях. В~дефектоскопии основной задачей является обнаружение различных включений в~однородном теле. Это позволяет проводить исследование различных объектов методом неразрушающего контроля.
\todo{вставить ссылки}

Однако аппроксимация реального первичного акустического поля
плоской волной справедлива только тогда, когда расстояние от~источника звука до рассеивателя много больше длины звуковой волны. На~практике это условие часто не~выполняется. В~этом случае нельзя не~учитывать криволинейность фронта падающей волны. Расходимость падающей волны приводит не~только к~количественным, но~и качественным изменениям дифракционной картины. Наибольший интерес представляет изучение дифракции звуковых волн, излучаемых цилиндрическими~и
сферическими источниками. С~помощью таких источников можно моделировать акустические поля сложных излучателей.


В~настоящей работе рассматривается задача о рассеянии цилиндрических звуковых волн упругой сферой, имеющей произвольно расположенную полость и радиально-неоднородное покрытие. С~помощью таких покрытий можно изменять звукоотражающие свойства тел. Такой слой возможно сделать в промышленных условиях, комбинируя несколько тонких однородных слоев с различными механическими характеристиками. В~качестве рассматриваемого тела выбрана сфера, т.к. более сложные тела можно аппроксиммировать с~помощью сферы. В ходе работы получено аналитическое описание акустического поля, рассеяного телом. Представлены результаты расчетов диаграмм направленности рассеянного поля.


\newpage
\section{Постановка задачи} 
%\rightsquigarrow, \circ\!\rightsquigarrow
\begin{table}[h]
\begin{tabular}{|l|c|c|c|c|}
\hline
 &Внешняя среда \scriptsize{O} &Полость \scriptsize{I} &Сфера \scriptsize{II}&Слой \scriptsize{III}\\
\hline
Координаты &\multicolumn{4}{|c|}{Прямоугольные}\\
\cline{2-5}
& --- & $x\c, y\c, z\c$ & $x\s, y\s, z\s$ & $x\s, y\s, z\s$\\
\cline{2-5}
&\multicolumn{4}{|c|}{Цилиндрические}\\
\cline{2-5}
&(ист.) $\r\en, \f\en, z\en$ & $\r\c, \f\c, z\c$ & $\r\s, \f\s, z\s$ & $\r\s, \f\s, z\s$\\
\cline{2-5}
&\multicolumn{4}{|c|}{Сферические}\\
\cline{2-5}
& --- & $r\c, \Q\c, \f\c$ & $r\s, \Q\s, \f\s$ & $r\s, \Q\s, \f\s$\\
\hline
Расстояние & $\hre, \hfe$ & $\hrc, \hfc$ & --- & ---\\
от сферы &&&&\\
\hline
Плотность &$p\en$ &$p\c$ &$p\s$ &$p\l$\\
\hline
Ак. давление &$P\en$ &$P\c$ & --- & ---\\
\hline
Скорость &$\bar{v}\en$ &$\bar{v}\c$ & --- & ---\\
\hline
Размер & --- & $R\c$ & $R\s$ & $R\l$\\
\hline
Скорость звука & $c\en$ & $c\c$ & ${c\s}_l, {c\s}_\t$ & ---\\
\hline
Потенциал & ист.: $\P_o (\text{\scriptsize{O}}\!\!: j\s)$ &$\P\c$ & $\P\s, \F\s$ & \\
скоростей&расс.: $\P_s(\text{\scriptsize{O}}\!\!: h\s)$& $(\text{\scriptsize{I}}\!\!: j\c)$ &$(\text{\scriptsize{II}}\!\!: j\s + h\c)$ & --- \\
волн&полн.: $\P\en (\text{\scriptsize{O}}\!\!: j\s + h\s)$&&&\\
\hline
Волновое число &$k\en$ &$k\c$ &$k\s$ &$k\l$\\
\hline
К-ты Ламе & --- & --- & $\la\s, \mu\s$ & $\la\l, \mu\l$\\
\hline
\end{tabular}
\caption{Обозначения, используемые в работе}
\end{table}

\newpage

Рассмотрим изотропный однородный упругий шар радиуса~$R\s$, плотность материала которого~$p\s$, упругие постоянные~$\la\s$ и~$\mu\s$, содержащий произвольно расположенную сферическую полость с~радиусом~$R\c$. Шар имеет покрытие в~виде неоднородного изотропного упругого слоя, внешний радиус которого равен~$R\l$.
 
Свяжем с полостью тела и с~самим телом прямоугольные системы координат $x\c, y\c, z\c$ и $x\s, y\s, z\s$ соответственно так, чтобы соответствующие оси обеих систем координат были параллельны. С~декартовыми системами координат $x\c, y\c, z\c$ и $x\s, y\s, z\s$ свяжем сферические координаты $r\c, \Q\c, \f\c$ и $r\s, \Q\s, \f\s$.

Полагаем, что модули упругости~$\la\l$ и~$\mu\l$ материала слоя описываются дифференцируемыми функциями радиальной координаты~$r\s$ сферической системы координат~$(r\s, \Q\s, \f\s)$, а плотность $p\l$~--- непрерывной функцией координаты~$r\s$.  Окружающая тело и находящаяся в его полости жидкости~--- идеальные и однородные, имеющие плотности~$p\en, p\c$ и скорости звука~$c\en, c\c$ соответственно. 

Определим отраженные от~тела и возбужденные в~его~полости волны, а также найдем поля смещений в~упругом материале шара и неоднородном слое.

\newpage
\section{Определение волновых полей}
Пусть из~внешнего пространства на~шар падает цилиндрическая звуковая волна, излучаемая бесконечно длинным линейным источником, который в цилиндрической системе координат $\r\s, \f\s, z\s$ с началом в центре шара имеет координаты $\r\s = \hre, \;\f\s = \hfe$ и параллелен оси $z\s.$

Потенциал скоростей гармонической звуковой волны, излучаемой цилиндрическим источником порядка $n,$ запишем в~виде:
\begin{equation}\label{potential speed}
\P_o(\r\en, \f\en, t) = A_o H_n(k\en\r\en)\exp\!\left[i\left(n\f\en-\om t\right)\right], \qq \text{где}
\end{equation}
$\r\en, \f\en, z\en$~--- цилиндрическая система координат, связанная с источником, оси которой одинаково ориентированы с осями координат $\r\s, \f\s, z\s$ рассеивателя.\\
$A_o$~--- амплитуда волны; \\
$H_n$~--- цилиндрическая функция Ханкеля первого рода;\\
$k\en = \om\en / c\en$~--- волновое число окружающей тело жидкости; \\
$\om\en$~--- круговая частота.\\
При этом координата $\r\en$ представляет собой расстояние от точки пространства до оси излучающего цилиндра. Оно также выражается через координаты цилиндрической системы координат, связанной с телом:
$$
\r\en = \left(\r\s^2+\hre^2-2\hre\r\s\cos(\f\s-\hfe)\right)^{1/2}.
$$

Рассмотрим монохроматическую симметричную цилиндрическую волну, которая излучается бесконечно длинным линейным источником, параллельным оси $z$. Она описывается с помощью цилиндрической функции Ханкеля первого рода нулевого порядка. Потенциал скоростей такой волны представляется в виде
$$
\P_o(\r\en, t) = A_oH_0(k\en\r\en)\e^{-i\om t}.
$$
В~дальнейшем временной множитель~$\exp(-i\om t)$ будем опускать.

Воспользовавшись теоремой сложения для цилиндрических функций Бесселя представим потенциал скоростей падающей волны в системе координат $\r\s, \f\s, z\s$ следующими разложениями:
$$
\P_o(\r\s, \f\s) = A_o \sum\limits_{m = 0}^\infty (2-\delta_{0m})\cos\bigl(m(\f\s-\hfe)\bigr) 
\begin{cases}
J_m(k\en\hre)H_m(k\en\r\s), & \r\s > \hre; \\
H_m(k\en\hre)J_m(k\en\r\s), & \r\s < \hre, \\
\end{cases}
$$
где $J_m$~--- цилиндрическая функция Бесселя первого рода порядка $m;$\\
$\delta_{0m}$~--- символ Кронекера.

Задача определения акустических полей вне~упругого тела и внутри его~полости в~установившемся режиме колебаний заключается в~нахождении решений уравнения Гельмгольца:
\begin{align}
\Lap\P\en &+ k\en^2\P\en = 0;\label{Helmholtz_ambient}\\
\Lap\P\c &+ k\c^2\P\c = 0,\label{Helmholtz_hollow}
\end{align}
где $\P\en$~--- потенциал скоростей полного акустического поля во~внешней среде;\\
$\P\c$~--- потенциал скоростей акустического поля в~полости тела;\\
$k\c = \df{\om}{c\c}$~--- волновое число находящейся в~полости жидкости.\\ При~этом скорости частиц жидкости и акустическое давление вне~тела и внутри полости определяются по~следующим формулам соответственно:
\begin{align}
\bar{v}\en &= \grad\P\en; &\q P\en &= ip\en\om\P\en;\\
\bar{v}\c &= \grad\P\c; &\q P\c &= ip\c\om\P\c.
\end{align}


В~силу линейной постановки задачи для $\P\en$ и $\P_o$ справедливо
\begin{equation} \label{potention_speed_ambient}
\P\en = \P_o + \P_s,		\qq\text{где}
\end{equation}
$\P_s$~--- потенциал скоростей рассеянной звуковой волны.\\
Тогда из~\eqref{Helmholtz_ambient} получаем уравнение для нахождения~$\P_s$:
\begin{equation} \label{Helmholtz_ambient_diffraction}
\Lap\P_s + k\en^2\P_s = 0.
\end{equation}

Из-за~произвольного расположения полости в теле потенциалы $\P\c$ и $\P_s$ не~будут проявлять свойства симметрии.
Уравнения~\cref{Helmholtz_hollow,Helmholtz_ambient_diffraction} запишем в~сферических системах координат~$(r\c, \Q\c, \f\c)$ и $(r\s, \Q\s, \f\s)$ соответственно:
\begin{align}
\fr1{r\c^2}\fr\de{\de r\c}\left(r\c^2 \fr{\de\P\c}{\de r\c}\right) + \fr1{r\c^2\sin^2\Q\c}\fr{\de\P\c}{\de\f\c^2} &+ \fr1{r\c^2\sin\Q\c}\fr\de{\de\Q\c} \left(\sin\Q\c \fr{\de\P\c}{\de\Q\c}\right) + k\c^2\P\c = 0;\\
\fr1{r\s^2}\fr\de{\de r\s}\left(r\s^2 \fr{\de\P_s}{\de r\s}\right) + \fr1{r\s^2\sin^2\Q\s}\fr{\de\P_s}{\de\f\s^2} &+ \fr1{r\s^2\sin\Q\s}\fr\de{\de\Q\s} \left(\sin\Q\s \fr{\de\P_s}{\de\Q\s}\right) + k\en^2\P_s = 0.
\end{align}

Звуковая волна в~полости тела~$\P\c$ должна удовлетворять условию ограниченности, а отраженная волна~$\P_s$~--- условиям излучения на~бесконечности. Поэтому потенциалы $\P_s$ и $\P\c$ будем искать в~виде 
\begin{align}\P_s(r\s, \Q\s, \f\s) = \sum\limits_{n = 0}^\infty \sum\limits_{m = 0}^n {A\en}_{nm} h_n(k\en r\s) P_n^m(\cos\Q\s)\cos\bigl(m(\f\s-\hfe)\bigr);\\
\P\c(r\c, \Q\c, \f\c) = \sum\limits_{n = 0}^\infty \sum\limits_{m = 0}^n {B\c}_{nm} j_n(k\c r\c) P_n^m(\cos\Q\c)\cos\bigl(m(\f\c-\hfc)\bigr),
\end{align}
где $h_n(x)$ и $j_n(x)$~--- сферические функции Ханкеля первого рода и Бесселя соответственно; \\
$P_n(x)$~--- многочлен Лежандра степени $n$.



\todo{continue...}


Распространение малых возмущений в~упругом теле для~установившегося режима движения частиц тела описывается скалярным и векторным уравнением Гельмгольца:
\begin{align}
\Lap\P\s &+ {k\s}_l^2\P\s = 0;\label{Helmholtz_scalar}\\
\Lap\F\s &+ {k\s}_\t^2\F\s = 0,\label{Helmholtz_vector}
\end{align}
где ${k\s}_l$~--- волновое число продольных волн со~скоростью распространения \break 
${c\s}_l = \sqrt{\df{(\la\s + 2\mu\s)}{p\s}}$;\\
${k\s}_\t$~--- волновое число поперечных волн со~скоростью распространения \\
${c\s}_\t = \sqrt{\df{\mu\s}{p\s}}$;\\
$\P\s$ и $\F\s$~--- скалярной и векторный потенциалы смещения соответственно.

Вектор смещения $\mathbf{u}\s$ частиц упругого тела определяется по~формуле
$$
\mathbf{u}\s = \grad \P\s + \rot \F\s.
$$

Потенциал смещения $\P\s$ будем искать в~виде ряда по~двум локальным сферическим функциям:
$$
\P\s = \sum\limits_{n=0}^\infty \sum\limits_{m=-n}^n {A\s}_{nm} h_n({k\s}_lr\c)P_n^m(\cos\Q\c)\e^{im\f\c} + {B\s}_{nm} j_n({k\s}_lr\s)P_n^m(\cos\Q\s)\e^{im\f\s}
$$

Векторный потенциал $\F\s$ может быть представлен в~виде суммы:
$$
\F\s = rV\er + \rot\bigl(rW\er\bigr),
$$
где $\er$~--- орт координатной оси~$r\s$ сферической системы координат~$r\s, \Q\s, \f\s,$\\
функции $V$ и $W$ удовлетворяют скалярным уравнениям Гельмгольца
\begin{align}
\Lap V &+ {k\s}_\t^2 V = 0,\\
\Lap W &+ {k\s}_\t^2 W = 0.
\end{align}

Функции $V$ и $W$ будем искать в~виде:
\begin{align}
V = \sum\limits_{n=0}^\infty \sum\limits_{m=-n}^n {C\s}_{nm} h_n({k\s}_lr\c)P_n^m(\cos\Q\c)\e^{im\f\c} + {D\s}_{nm} j_n({k\s}_lr\s)P_n^m(\cos\Q\s)\e^{im\f\s},\\
W = \sum\limits_{n=0}^\infty \sum\limits_{m=-n}^n {E\s}_{nm} h_n({k\s}_lr\c)P_n^m(\cos\Q\c)\e^{im\f\c} + {F\s}_{nm} j_n({k\s}_lr\s)P_n^m(\cos\Q\s)\e^{im\f\s}.
\end{align}

Коэффициенты разложений ${A\c}_{nm}, {B\en}_{nm}, {A\s}_{nm}, {B\s}_{nm}, {C\s}_{nm}, {D\s}_{nm}, {E\s}_{nm}$ и $ {F\s}_{nm}$ подлежат определению из граничных условий, которые заключаются в равенстве нормальных скоростей частиц упругой среды и жидкости на внешней поверхности слоя и внутренней поверхности полого шара; равенстве на них нормального напряжения и акустического давления; отсутствии на этих поверхностях касательных напряжений. На внутренней поверхности слоя при переходе через границу раздела упругих сред должны быть непрерывны составляющие вектора смещения частиц, а также нормальные и тангенциальные напряжения. Имеем:

\begin{equation*}
\begin{aligned}
\text{при }r\c &= R\c: \q  &  \si_{rr} &= -P_1;  &  \si_{r\Q} &= 0;  &  \si_{r\f} &= 0; &  -i\om u_r &= v_{1r};\\
\text{при }r_2 &= R_2: \q  &  \si_{rr} &= -P_2;  &  \si_{r\Q} &= 0;  &  \si_{r\f} &= 0; &  -i\om u_r &= v_{2r};\\
%
%-----------------------------------------------
%\text{при }r_2 &= R_3: \q  &  \si_{rr} &= -P_1;  &  \si_{r\Q} &= 0;  &  \si_{r\f} &= 0; &  -i\om u_r &= v_{1r};
\end{aligned}
\end{equation*}
 
\newpage
\section*{ЗАКЛЮЧЕНИЕ}
\addcontentsline{toc}{section}{ЗАКЛЮЧЕНИЕ}
Была поставлена задача о~дифракции цилиндрических звуковых волн на~упругой сфере, имеющей произвольно расположенную полость и неоднородное покрытие. В~данной работе приведены основные уравнения колебаний, а также разложения в~ряд искомых функций для~внешней среды сферы, а также полости тела.

\end{document}