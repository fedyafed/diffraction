\input{top.tex}

\renewcommand{\bibname}{СПИСОК ИСПОЛЬЗОВАННЫХ ИСТОЧНИКОВ}
\renewcommand\refname{СПИСОК ИСПОЛЬЗОВАННЫХ ИСТОЧНИКОВ}

%\input{title.tex}

%\newpage
\setcounter{page}{5}
\thispagestyle {empty}
\renewcommand{\contentsname}{\centering СОДЕРЖАНИЕ}
\tableofcontents

\newpage
\section*{ВВЕДЕНИЕ}
\addcontentsline{toc}{section}{ВВЕДЕНИЕ}
Потребность в~изучении дифракции на~различных телах очень высока. Знания, полученные путем изучения дифракции с~помощью моделей, используются как в~гидроакустике и эхолокации, так и в~других областях. В~дефектоскопии основной задачей является обнаружение различных включений в~однородном теле. Это позволяет проводить исследование различных объектов методом неразрушающего контроля.

В~настоящей работе рассматривается задача дифракции плоских звуковых волн на~упругой сфере, имеющей произвольно расположенную полость и неоднородное покрытие. С~помощью таких покрытий можно изменять звукоотражающие свойства тел. В~качестве рассматриваемого тела выбрана сфера, т.к. более сложные тела можно аппроксиммировать с~помощью сферы. 


\newpage

\section{Постановка задачи.} Рассмотрим изотропный однородный упругий шар радиуса~$r_0$, плотность материала которого~$\r_0$, упругие постоянные~$\la_0$ и~$\mu_0$. Шар имеет покрытие в~виде неоднородного изотропного упругого слоя, внешний радиус которого равен~$r_1$. Полагаем, что модули упругости~$\la_1$ и~$\mu_1$ материала слоя описываются дифференцируемыми функциями радиальной координаты~$r$ сферической системы координат~$(r, \Q, \f)$, а плотность $\r_1$~--- непрерывной функцией координаты~$r$. Окружающая тело жидкость~--- идеальная, ее плотность~$\r_2$, скорость звука~$c$. 

Пусть из~внешнего пространства на~шар падает плоская звуковая волна. Потенциал скоростей гармонической падающей волны запишем в~виде:
\begin{equation}\label{potential speed}
\P_2(\bar{\mathbf{x}}, t) = A_2 \exp\left[i\left(\bar{\mathbf{k}}\dotm\bar{\mathbf{x}}-\om t\right)\right],
\end{equation}
где $A_2$~--- амплитуда волны, $\bar{\mathbf{k}}$~--- волновой вектор в~окружающей жидкости,  $\lvert\bar{\mathbf{k}}\rvert = k = \om / c$~--- волновое число, $\om$~--- круговая частота.

 Без~ограничения общности будем полагать, что волна распространяется в~направлении~$\Q = 0$. Тогда в~сферической системе координат \eqref{potential speed} запишется в~виде:
\begin{equation}
\P_2(r, \Q, t) = A_2 \exp\left[i\left(kr\cos\Q - \om t\right)\right],
\end{equation}
В~дальнейшем временной множитель $\exp(-i\om t)$ будем опускать.

Определим отраженную от тела волну, а также найдем поля смещений в~упругом шаре и неоднородном слое.



\end{document}