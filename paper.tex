\input{top.tex}

\renewcommand{\bibname}{СПИСОК ИСПОЛЬЗОВАННЫХ ИСТОЧНИКОВ}
\renewcommand\refname{СПИСОК ИСПОЛЬЗОВАННЫХ ИСТОЧНИКОВ}

%\input{title.tex}

%\newpage
\setcounter{page}{5}
\thispagestyle {empty}
\renewcommand{\contentsname}{\centering СОДЕРЖАНИЕ}
\tableofcontents

\newpage
\section*{ВВЕДЕНИЕ}
\addcontentsline{toc}{section}{ВВЕДЕНИЕ}
Потребность в~изучении дифракции на~различных телах очень высока. Знания, полученные путем изучения дифракции с~помощью моделей, используются как в~гидроакустике и эхолокации, так и в~других областях. В~дефектоскопии основной задачей является обнаружение различных включений в~однородном теле. Это позволяет проводить исследование различных объектов методом неразрушающего контроля.

В~настоящей работе рассматривается задача дифракции плоских звуковых волн на~упругой сфере, имеющей произвольно расположенную полость и неоднородное покрытие. С~помощью таких покрытий можно изменять звукоотражающие свойства тел. В~качестве рассматриваемого тела выбрана сфера, т.к. более сложные тела можно аппроксиммировать с~помощью сферы. 


\newpage

\section{Постановка задачи.} Рассмотрим изотропный однородный упругий шар радиуса~$r_2$, плотность материала которого~$\r_2$, упругие постоянные~$\la_2$ и~$\mu_2$, содержащий произвольно расположенную сферическую полость с~радиусом~$r_1$. Шар имеет покрытие в~виде неоднородного изотропного упругого слоя, внешний радиус которого равен~$r_3$. Полагаем, что модули упругости~$\la_3$ и~$\mu_3$ материала слоя описываются дифференцируемыми функциями радиальной координаты~$r$ сферической системы координат~$(r, \Q, \f)$, а плотность $\r_3$~--- непрерывной функцией координаты~$r$.  Окружающая тело и находящаяся в его полости жидкости~--- идеальные и однородные, имеющие плотности~$\r_0, \r_1$ и скорости звука~$c_0, c_1$ соответственно. 

Свяжем с полостью тела и с~самим телом прямоугольные системы координат $x_1, y_1, z_1$ и $x_2, y_2, z_2$ соответственно так, чтобы соответствующие оси обеих систем координат были параллельны. С~декартовыми системами координат $x_1, y_1, z_1$ и $x_2, y_2, z_2$ свяжем сферические координаты $r_1, \Q_1, \f_1$ и $r_2, \Q_2, \f_2$.

Пусть из~внешнего пространства на~шар падает плоская звуковая волна. Потенциал скоростей гармонической падающей волны запишем в~виде:
\begin{equation}\label{potential speed}
\P_0(\bar{\mathbf{x}}, t) = A_0 \exp\left[i\left(\bar{\mathbf{k}}\dotm\bar{\mathbf{x}}-\om t\right)\right],
\end{equation}
где $A_0$~--- амплитуда волны, $\bar{\mathbf{k}}$~--- волновой вектор в~окружающей жидкости,  $\lvert\bar{\mathbf{k}}\rvert = k = \om / c$~--- волновое число, $\bar{\mathbf{x}}$~--- радиус-вектор, $\om$~--- круговая частота.

 Без~ограничения общности будем полагать, что волна распространяется в~направлении~$\Q_1 = \Q_2 = 0$. Тогда в~сферической системе координат \eqref{potential speed} запишется в~виде:
\begin{equation}
\P_0(r, \Q, t) = A_0 \exp\left[i\left(kr\cos\Q - \om t\right)\right],
\end{equation}
В~дальнейшем временной множитель~$\exp(-i\om t)$ будем опускать.

Определим отраженную от~тела и возбужденные в~его~полости волны, а также найдем поля смещений в~упругом материале шара и неоднородном слое.



\end{document}