\input{top.tex}

\renewcommand{\bibname}{СПИСОК ИСПОЛЬЗОВАННЫХ ИСТОЧНИКОВ}
\renewcommand\refname{СПИСОК ИСПОЛЬЗОВАННЫХ ИСТОЧНИКОВ}

%\input{title.tex}

%\newpage
\setcounter{page}{2}
\thispagestyle {empty}
\renewcommand{\contentsname}{\centering СОДЕРЖАНИЕ}
\tableofcontents


\newpage
\section*{ВВЕДЕНИЕ}
\addcontentsline{toc}{section}{ВВЕДЕНИЕ}
Потребность в изучении дифракции на различных телах очень высока. Знания, полученные путем изучения дифракции с помощью моделей, используются как в гидроакустике и эхолокации, так и в других областях. Для современного мира изучение простых моделей уже не дает требуемой точности прогнозирования поведения волн. Поэтому, необходимо изучать более сложные модели, детально описывающие рассматриваемые тела и окружающую среду.
В качестве простых моделей рассматривают дифракцию плоских звуковых волн, однако это приближение возможно только в случае, когда расстояние от источника до рассеивателя много больше длины волны. Акустические поля сложных излучателей успешнее моделируются при помощи изучения дифракции звуковых волн, излучаемых цилиндрическими и сферическими источниками.

В настоящей работе рассматриваются основные уравнения, используемые для простроения модели в задаче о дифракции звуковых волн на упругой сфере с неоднородным покрытием и неконцентрической полостью. С помощью таких покрытий можно изменять звукоотражающие свойства тел. В качестве рассматриваемого тела выбрана сфера, т.к. более сложные тела можно аппроксиммировать с~помощью сферы. Данная задача может быть рассмотрена с точки зрения дефектоскопии из-за наличия произвольно расположенной полости внутри тела.


\newpage
%\section{ДИФРАКЦИЯ ЗВУКОВЫХ ВОЛН НА УПРУГИХ ТЕЛАХ}
%\subsection{Обзор литературы}
\section{ОБЗОР ЛИТЕРАТУРЫ}
Изучению  распространения и дифракции звуковых волн в непрерывно-не\-од\-но\-род\-ных средах посвящено большое количество работ. При этом большая часть исследований относится к рассмотрению дифракции звука на жидких неоднородных телах. Проблема дифракции звуковых волн на упругих неоднородных телах является значительно более сложной по сравнению с аналогичной проблемой для неоднородных акустических сред.

Дифракция плоских звуковых волн на упругих сферических и цилиндрических телах рассматривалось в ряде работ. 
В~\cite{Faran} упругая сплошная сфера полагалась однородной и изотропной.
В~\cite{transversely isotropic spherical layer} полый сферический рассеиватель являлся неоднородным и трансверсально-изотропным. 
В~\cite{inhomogeneous thermo elastic spherical layer} изучалась дифракция на полой неоднородной термоупругой сфере.
В работе~\cite{plane elastic wave inhomogeneous cylinder} рассмотрена задача о рассеянии плоской продольной упругой волны цилиндром с неоднородным внешним слоем. Предполагается, что падающая волна распространяется перпендикулярно к образующей цилиндра, а материал внутренней части цилиндра является однородной изотропной упругой средой.
В~\cite{plane elastic wave inhomogeneous ball} найдено решение задачи о рассеянии плоской упругой волны однородным изотропным шаром, окруженным неоднородным сферическим слоем.


Во всех указанных работах неоднородные упругие среды полагались изотропными, жидкость в которую помещены тела, полагалась идеальной, а также что рассеиватель находится в безграничной среде.

\newpage
\section{МАТЕМАТИЧЕСКОЕ МОДЕЛИРОВАНИЕ РАСПРОСТРАНЕНИЯ ЗВУКА}
\subsection{Распространение звука в идеальной жидкости}

Для математического процесса моделирования распространения звука в идеальной среде воспользуемся полной системой уравнений гидромеханики идеальной жидкости, описывающей любые движения идеальной жидкости. Эта система включает уравнение движения идеальной жидкости (уравнение Эйлера), уравнение неразрывности и уравнение физического состояния.

Математическое описание движения жидкости осуществляется с помощью функций, определяющих распределение скорости $\bar{v}$, давления $p$ и плотности $\rho$. Уравнение Эйлера имеет вид
\begin{equation}\label{eq1}
\fr{\de \bar{v}}{\de t} + (\bar{v} \cdot \nab)\bar{v} = \bar F - \fr{1}{\rho}\grad p,
\end{equation}
где $\bar F$ --- массовая сила, отнесённая к единице массы.

Уравнение неразрывности записывается в виде
\begin{equation}\label{eq2}
\fr{\de \bar{\rho}}{\de t} + \diver (\rho \bar{v}) = 0.
\end{equation}

Будем считать, что движение сжимаемой жидкости происходит адиабатически. В этом случае уравнение физического состояния принимает вид
\begin{equation}\label{eq3}
p=p_0\bigg(\fr{\rho}{\rho_0}\bigg)^\gamma,
\end{equation}
где $\gamma=\fr{C_p}{C_{\nu}}; \; p_0$ и $\rho_0$ --- давление и плотность невозмущенной жидкости; \\ $C_p$ и $C_{\nu}$ --- теплоемкость при постоянном давлении и постоянном объеме.

Процесс распространения звука представляет собой малые колебания жидкости, так что в уравнении \eqref{eq1} можно пренебречь конвективными членами. Полагая, что внешние силы отсутствуют, получим:
\begin{equation}\label{eq4}
\fr{\de \bar{v}}{\de t} = - \fr{1}{\rho}\grad p.
\end{equation}
Введем в рассмотрение величину $s$, называемую сжатием и равную относительному изменению плотности
$$
s=\fr{\rho - \rho_0}{\rho_0}.
$$
Тогда
\begin{equation}\label{eq5}
\rho = \rho_0 (1+s).
\end{equation}

Тогда уравнение \eqref{eq3} перепишем в виде
\begin{equation}\label{eq6}
p=p_0(1+s)^\gamma.
\end{equation}
При малых колебаниях жидкости сжатие $s$ настолько мало, что высшими степенями  $s$ можно пренебречь. В результате из выражения \eqref{eq6} получим
\begin{equation}\label{eq7}
p=p_0(1+\gamma \cdot s).
\end{equation}

Подставим выражение \eqref{eq5} в уравнение неразрывности. Так как 
$$
\diver(\rho \bar{v})= \rho\diver(\bar{v})+\bar{v} \grad \rho = \rho_0 \diver(\bar{v}) + \rho_0 s \diver(\bar{v}) + \bar{v} \grad \rho,
$$
причем последними двумя слагаемыми можно пренебречь, то вместо уравнения \eqref{eq2} будем иметь
\begin{equation}\label{eq8}
\fr{\de s}{\de t} + \diver (\bar{v}) = 0.
\end{equation}
Уравнение \eqref{eq4} в том же приближении сводится к уравнению
\begin{equation}\label{eq9}
\fr{\de \bar{v}}{\de t} = -c^2\cdot\grad s,
\end{equation}
где $c=(\gamma \fr{p_0}{\rho_0})^{1/2}$ --- скорость звука.

Предположим теперь, что в начальный момент существует потенциал скоростей $\tilde{\psi_0}$, т.е.
\begin{equation}\label{eq 10}
\bar{v}\big|_{t=0}=\grad\tilde{\psi_0}.
\end{equation}

Из уравнения \eqref{eq9} имеем
$$
\bar{v}=\bar{v}|_{t=0}-c^2\grad \bigg(\int\limits_{0}^{t} s\diff t\bigg).
$$
С учетом \eqref{eq 10} получаем
\begin{equation}\label{eq 11}
\bar{v}=\grad\bigg[\tilde{\psi_0}-c^2\int\limits_{0}^{t} s\diff t\bigg]=\grad\tilde{\psi}.
\end{equation}
Это означает, что существует потенциал скоростей $\tilde{\psi}$ в любой момент времени $t:$
$$
\tilde{\psi}=\tilde{\psi_0}-c^2\int\limits_{0}^{t} s\diff t.
$$

Дифференцируя последнее выражение два раза по $t,$ получим
\begin{equation}\label{eq 12}
\fr{\de^2\tilde{\psi}}{\de t^2}=-c^2\fr{\de s}{\de t}.
\end{equation}

С другой стороны, подставляя выражение \eqref{eq 11} в уравнение \eqref{eq8}, будем иметь
\begin{equation}\label{eq 13}
\fr{\de s}{\de t}=-\diver\grad\tilde{\psi}=-\Lap\tilde{\psi}.
\end{equation}
Из уравнений \eqref{eq 12} и \eqref{eq 13} приходим к волновому уравнению
\begin{equation}\label{eq 14}
\fr{\de^2\tilde{\psi}}{\de t^2}=c^2\Lap\tilde{\psi},
\end{equation}
которое описывает процесс распространения звука в идеальной жидкости.

Отметим, что знания потенциала $\tilde{\psi}$ достаточно для определения всего процесса движения жидкости в случае малых возмущений, так как
$$
\bar{v}=\grad\tilde{\psi}\text{;}\quad s=-\fr{1}{c^2}\fr{\de\tilde{\psi}}{\de t}.
$$

Найдем акустическое давление $p'=p-p_0$. Из уравнения \eqref{eq4}, используя приближение, сделанное для \eqref{eq9}:
$$
\fr{\de \grad \tilde\psi}{\de t} = - \fr{1}{\rho_0}\grad p'.
$$

Занесем дифференциал под градиент и перенесем в левую часть:
$$
\grad \left(\fr{\de \tilde\psi}{\de t} + \fr{p'}{\rho_0}\right) = 0.
$$
Выражение в скобках не зависит от координат. Учитывая, что потенциал $\tilde\psi$ определяется с точностью до функции времени, приравняем выражение в скобках нулю:
$$
\fr{\de \tilde\psi}{\de t} + \fr{p'}{\rho_0} = 0.
$$

В случае установившегося режима колебаний
\begin{equation}\label{eq 15}
\tilde{\psi}=\psi \e^{-i\omega t}
\end{equation}
уравнение \eqref{eq 14} переходит в уравнение Гельмгольца
\begin{equation}\label{eq 16}
\Lap{\psi}+k^2\psi=0,
\end{equation}
где $\omega$ --- круговая частота; $k=\fr{\omega}{c}$ --- волновое число.
При этом акустическое давление 
$$
p'=-\rho_0 \fr{\de \tilde\psi}{\de t}=-\rho_0 \cdot (-i \omega) \psi \e^{-i\omega t}=i\rho_0\omega\tilde{\psi}.
$$


\newpage
\subsection{Распространение малых возмущений в упругой среде}
Под идеально упругим телом понимают тело, которое под воздействием приложенных к нему сил деформируется и полностью восстанавливает свою форму после устранения причины, вызвавшей деформацию. В качестве рассматриваемой среды возьмем модель линейной неоднородной изотропной упругой среды.

Уравнения движения сплошной среды, для случая отсутствия массовых сил, в криволинейной ортогональной системе координат имеют вид~\cite{Nowacki}:
\begin{equation}\label{eq moving}
\begin{split}
\fr{1}{h_1h_2h_3}\biggl[\fr{\de}{\de q_1} \biggl(h_2h_3\s_{11}\biggr) + \fr{\de}{\de q_2} \biggl(h_1h_3\s_{12}\biggr) +\fr{\de}{\de q_3} \biggl(h_1h_2\s_{13}\biggr) -\\
 -\s_{22}h_3\fr{\de h_2}{\de q_1} -\s_{33}h_2\fr{\de h_3}{\de q_1} + \s_{12}h_3\fr{\de h_1}{\de q_2} + \s_{13}h_2\fr{\de h_1}{\de q_3}\biggr] = \rho \fr{\de^2 u_1}{\de t^2};\\
 \fr{1}{h_1h_2h_3}\biggl[\fr{\de}{\de q_2} \biggl(h_1h_3\s_{22}\biggr) + \fr{\de}{\de q_3} \biggl(h_1h_2\s_{23}\biggr) +\fr{\de}{\de q_1} \biggl(h_2h_3\s_{12}\biggr) -\\
 -\s_{33}h_1\fr{\de h_3}{\de q_2} -\s_{11}h_3\fr{\de h_1}{\de q_2} + \s_{23}h_1\fr{\de h_2}{\de q_3} + \s_{12}h_3\fr{\de h_2}{\de q_1}\biggr] = \rho \fr{\de^2 u_2}{\de t^2};\\
 \fr{1}{h_1h_2h_3}\biggl[\fr{\de}{\de q_3} \biggl(h_1h_2\s_{33}\biggr) + \fr{\de}{\de q_1} \biggl(h_2h_3\s_{13}\biggr) +\fr{\de}{\de q_2} \biggl(h_1h_3\s_{23}\biggr) -\\
 -\s_{11}h_2\fr{\de h_1}{\de q_3} -\s_{22}h_1\fr{\de h_2}{\de q_3} + \s_{13}h_2\fr{\de h_3}{\de q_1} + \s_{23}h_1\fr{\de h_3}{\de q_2}\biggr] = \rho \fr{\de^2 u_3}{\de t^2}.\
\end{split}
\end{equation}
Здесь $\rho = \rho(\textbf{r})$ --- равновесная плотность среды ($\textbf{r}$ --- радиус-вектор точки тела);\\
$u_i$ --- компоненты вектора смещений $u;$\\
$\s_{ij}$ --- компоненты тензора напряжений;\\
$h_1,h_2,h_3$ --- коэффциенты Ламе криволинейной системы координат.

Обобщенный закон Гука --- связывает тензоры напряжения и деформации с помощью тензора упругих постоянных~\cite{Nowacki}:
\begin{equation}\label{Hook}
\s_{q_iq_j}=C_{ijkl}\eps_{q_kq_l}.
\end{equation}

В силу симметрии тензоров напряжении и деформации следующие компоненты тензора $C$ будут равны:
\begin{equation*}
\begin{split}
C_{ijkl}&=C_{ijlk};\\
C_{ijkl}&=C_{jilk};\\
C_{ijkl}&=C_{klij}.\\
\end{split}
\end{equation*}

Для случая неоднородного изотропного упругого тела модули упругости могут быть выражены через два независимых модуля упругости Ламе $\la, \mu:$
$$
C_{ijkl} = \mu(\delta_{ik}\delta_{jl}+\delta_{il}\delta_{jk}) + \la\delta_{ij}\delta{kl}.
$$
Подставляя в уравнение \eqref{Hook}, получим:
\begin{equation}\label{Hook2}
\s_{ij}=2\mu\eps_{ij}+\la\delta_{ij}\eps_{kk}.
\end{equation}

Компоненты тензора деформаций связаны с компонентами вектора смещения в ортогональной криволинейной системе координат $q_1,q_2,q_3$ следующими соотношениями~\cite{Nowacki}:
\begin{equation}\label{eq deform}
\begin{split}
\eps_{11} &= \fr{1}{h_1}\fr{\de u_1}{\de q_1} + \fr{1}{h_1h_2}\fr{\de h_1}{\de q_2}u_2 + \fr{1}{h_1h_3}\fr{\de h_1}{\de q_3}u_3;\\
\eps_{22} &= \fr{1}{h_2}\fr{\de u_2}{\de q_2} + \fr{1}{h_2h_3}\fr{\de h_2}{\de q_3}u_3 + \fr{1}{h_1h_2}\fr{\de h_2}{\de q_1}u_1;\\
\eps_{33} &= \fr{1}{h_3}\fr{\de u_3}{\de q_3} + \fr{1}{h_1h_3}\fr{\de h_3}{\de q_1}u_1 + \fr{1}{h_2h_3}\fr{\de h_3}{\de q_2}u_2;\\
\eps_{12} &= \fr12\left[\fr{h_1}{h_2}\fr{\de}{\de q_2}\left(\fr{u_1}{h_1}\right) + \fr{h_2}{h_1}\fr{\de}{\de q_1}\left(\fr{u_2}{h_2}\right) \right];\\
\eps_{13} &= \fr12\left[\fr{h_1}{h_3}\fr{\de}{\de q_3}\left(\fr{u_1}{h_1}\right) + \fr{h_3}{h_1}\fr{\de}{\de q_1}\left(\fr{u_3}{h_3}\right) \right];\\
\eps_{23} &= \fr12\left[\fr{h_2}{h_3}\fr{\de}{\de q_3}\left(\fr{u_2}{h_2}\right) + \fr{h_3}{h_2}\fr{\de}{\de q_2}\left(\fr{u_3}{h_3}\right) \right].\\
\end{split}
\end{equation}

Таким образом, математическая модель, описывающая распространение малых возмущений в изотропной неоднородной упругой среде состоит из уравнений движения сплошной среды \eqref{eq moving}, закона Гука \eqref{Hook2}, выражения компонентов тензора деформаций через компоненты вектора смещения \eqref{eq deform}, начальных и граничных условий.


\newpage

\section*{ЗАКЛЮЧЕНИЕ}
\addcontentsline{toc}{section}{ЗАКЛЮЧЕНИЕ}

В настоящей работе рассматриваются основные уравнения, используемые для простроения модели в задаче о дифракции звуковых волн на упругой сфере с неоднородным покрытием и неконцентрической полостью. Взяв за основу уравнение Эйлера и уравнение неразрывности, при условии малых колебаний, было получено уравнение Гельмгольца для потенциала скорости. Для упругой среды в качестве основных уравнений были представлены уравнения движения сплошной среды, уравнения, олицетворяющие закон Гука и выражения компонентов тензора деформаций через компоненты вектора смещения. 



\newpage

\begin{thebibliography}{99}

\bibitem{Faran}
Faran~J.\,J. Sound scattering by solid cylinders and spheres // Acoust. Soc. Amer. 1951. V.~23. №~4. P.~405--420.

\bibitem{transversely isotropic spherical layer}
Скобельцын~С.\,А., Толоконников~Л.\,А. Рассеяние звука неоднородным трансверсально-изотропным сферическим слоем // Акуст. журн. 1995. Т.~41. Вып.~6. С.~917--923.

\bibitem{inhomogeneous thermo elastic spherical layer}
Ларин~Н.\,В., Толоконников~Л.\,А. Рассеяние звука неоднородным термоупругим сферическим слоем // Прикладная математика и механика. 2010. Т.~74. Вып.~4. С.~645--654.

\bibitem{plane elastic wave inhomogeneous cylinder}
Скобельцын~С.\,А. Метод конечных элементов в задаче о рассеянии плоской упругой волны неоднородным цилиндром // Известия ТулГУ. Серия Математика. Механика. Информатика. 2005. Т.~11. Вып.~5. С.~187--200.

\bibitem{plane elastic wave inhomogeneous ball}
Авдеев~И.\,С., Скобельцын~С.\,А. Дифракция плоской упругой волны на неоднородном шаре // Известия ТулГУ. Серия Геодинамика, физика, математика, термодинамика, геоэкология. 2006. Вып.~3. С.~138--139.

%\bibitem{cylindrical elastic sphere}
%Толоконников~Л.\,А. Дифракция цилиндрических звуковых волн на упругой сфере // Дифференциальные уравнения и прикладные задачи. --- Тула: Изд-во ТулГТУ, 1995. С.~82--86.

%\bibitem{inhomogeneous thermoelastic bodies}
%Толоконников~Л.\,А., Ларин~Н.\,В. Рассеяние звука неоднородными термоупругими телами. --- Тула: Изд-во ТулГУ, 2008. 232~с.

%\bibitem{elastic cylindrical waves inhomogeneous cylinder}
%Скобельцын~С.\,А. Задача о рассеянии упругих цилиндрических волн неоднородным цилиндром // Известия ТулГУ. Серия Геодинамика, физика, математика, термодинамика, геоэкология. 2006. Вып.~3. С.~126--138.

\bibitem{Nowacki}
Новацкий~В. Теория упругости. Т.~2. --- М.: Мир, 1975. 872~с.

%\bibitem{Shenderov}
%Шендеров~Е.\,Л. Волновые задачи гидроакустики. --- Л.: Судостроение, 1972. 352~с.

%\bibitem{Mors Feshbach}
%Морс~Ф.\,М., Фешбах~Г. Методы теоретической физики. Т.~2. --- М.: Изд-во иностр. лит., 1960. 886~с.

%\bibitem{Korn}
%Корн~Г., Корн~Т. Справочник по математике. --- М.: Наука, 1968. 720~с.

%\bibitem{Lebedev}
%Лебедев~Н.\,Н. Специальные функции и их приложения. --- М.: Физматгиз, 1963. 358~с.







%%\bibitem{Sommerfeld}
%%Зоммерфельд~А. Дифференциальные уравнения в частных производных физики. --- М.: Изд-во иностр. лит., 1950. С.~154--157.
\end{thebibliography}

\end{document}