\input{top.tex}

\renewcommand{\bibname}{СПИСОК ИСПОЛЬЗОВАННЫХ ИСТОЧНИКОВ}
\renewcommand\refname{СПИСОК ИСПОЛЬЗОВАННЫХ ИСТОЧНИКОВ}

%\input{title.tex}

%\newpage
\setcounter{page}{5}
\thispagestyle {empty}
\renewcommand{\contentsname}{\centering СОДЕРЖАНИЕ}
\tableofcontents

\newpage
\section*{ВВЕДЕНИЕ}
\addcontentsline{toc}{section}{ВВЕДЕНИЕ}
Потребность в изучении дифракции на различных телах очень высока. Знания, полученные путем изучения дифракции с помощью моделей, используются как в гидроакустике и эхолокации, так и в других областях. Для современного мира изучение простых моделей уже не дает требуемой точности прогнозирования поведения волн. Поэтому, необходимо изучать более сложные модели, детально описывающие рассматриваемые тела и окружающую среду.
В качестве простых моделей рассматривают дифракцию плоских звуковых волн, однако это приближение возможно только в случае, когда расстояние от источника до рассеивателя много больше длины волны. Акустические поля сложных излучателей успешнее моделируются при помощи изучения дифракции звуковых волн, излучаемых цилиндрическими и сферическими источниками.

В настоящей работе рассматривается задача дифракции цилиндрических звуковых волн на упругой сфере с неоднородным покрытием. С помощью таких покрытий можно изменять звукоотражающие свойства тел. В качестве рассматриваемого тела выбрана сфера, т.к. более сложные тела можно аппроксиммировать до сферы. 


\newpage
\section{ДИФРАКЦИЯ ЗВУКОВЫХ ВОЛН НА УПРУГИХ ТЕЛАХ}
\subsection{Обзор литературы}
Изучению  распространения и дифракции звуковых волн в непрерывно-не\-од\-но\-род\-ных средах посвящено большое количество работ. При этом большая часть исследований относится к рассмотрению дифракции звука на жидких неоднородных телах. Проблема дифракции звуковых волн на упругих неоднородных телах является значительно более сложной по сравнению с аналогичной проблемой для неоднородных акустических сред.

Дифракция плоских звуковых волн на упругих сферических и цилиндрических телах рассматривалось в ряде работ. 
В~\cite{Faran} упругая сплошная сфера полагалась однородной и изотропной.
В~\cite{transversely isotropic spherical layer} полый сферический рассеиватель являлся неоднородным и трансверсально-изотропным. 
В~\cite{inhomogeneous thermo elastic spherical layer} изучалась дифракция на полой неоднородной термоупругой сфере.
В работе~\cite{plane elastic wave inhomogeneous cylinder} рассмотрена задача о рассеянии плоской продольной упругой волны цилиндром с неоднородным внешним слоем. Предполагается, что падающая волна распространяется перпендикулярно к образующей цилиндра, а материал внутренней части цилиндра является однородной изотропной упругой средой.
В~\cite{plane elastic wave inhomogeneous ball} найдено решение задачи о рассеянии плоской упругой волны однородным изотропным шаром, окруженным неоднородным сферическим слоем.

Однако аппроксимация реального первичного акустического поля плоской волной справедлива только тогда, когда расстояние от источника звука до рассеивателя много больше длины звуковой волны. На практике это условие часто не выполняется. В этом случае нельзя не учитывать криволинейность фронта падающей волны. Расходимость падающей волны приводит не только к количественным, но и качественным изменениям дифракционной картины. Наибольший интерес представляет изучение дифракции звуковых волн, излучаемых цилиндрическими и сферическими источниками. С помощью таких источников можно моделировать акустические поля сложных излучателей.
Задача дифракции цилиндрических звуковых волн на упругой однородной сфере была решена в~\cite{cylindrical elastic sphere}. 
Дифракция цилиндрических и сферических звуковых волн на неоднородном термоупругом полом шаре исследовалась в~\cite{inhomogeneous thermoelastic bodies}.
Рассеяние упругих цилиндрических продольных волн упругим однородным цилиндром с неоднородным внешним слоем рассмотрено в~\cite{elastic cylindrical waves inhomogeneous cylinder}.

Во всех указанных работах неоднородные упругие среды полагались изотропными, жидкость в которую помещены тела, полагалась идеальной, а также что рассеиватель находится в безграничной среде.

\newpage
\subsection{Математическое моделирование распространения звука}
\subsubsection{Распространение звука в идеальной жидкости}

Для математического процесса моделирования распространения звука в идеальной среде воспользуемся полной системой уравнений гидромеханики идеальной жидкости, описывающей любые движения идеальной жидкости. Эта система включает уравнение движения идеальной жидкости (уравнение Эйлера), уравнение неразрывности и уравнение физического состояния.

Математическое описание движения жидкости осуществляется с помощью функций, определяющих распределение скорости $\bar{v}$, давления $p$ и плотности $\rho$. Уравнение Эйлера имеет вид
\begin{equation}\label{eq1}
\fr{\de \bar{v}}{\de t} + (\bar{v} \cdot \nab)\bar{v} = \bar F - \fr{1}{\rho}\grad p,
\end{equation}
где $\bar F$ --- массовая сила, отнесённая к единице массы.

Уравнение неразрывности записывается в виде
\begin{equation}\label{eq2}
\fr{\de \bar{\rho}}{\de t} + \diver (\rho \bar{v}) = 0.
\end{equation}

Будем считать, что движение сжимаемой жидкости происходит адиабатически. В этом случае уравнение физического состояния принимает вид
\begin{equation}\label{eq3}
p=p_0\bigg(\fr{\rho}{\rho_0}\bigg)^\gamma,
\end{equation}
где $\gamma=\fr{C_p}{C_{\nu}}; \; p_0$ и $\rho_0$ --- давление и плотность невозмущенной жидкости; \\ $C_p$ и $C_{\nu}$ --- теплоемкость при постоянном давлении и постоянном объеме.

Процесс распространения звука представляет собой малые колебания жидкости, так что в уравнении \eqref{eq1} можно пренебречь конвективными членами. Полагая, что внешние силы отсутствуют, получим:
\begin{equation}\label{eq4}
\fr{\de \bar{v}}{\de t} = - \fr{1}{\rho}\grad p.
\end{equation}
Введем в рассмотрение величину $s$, называемую сжатием и равную относительному изменению плотности
$$
s=\fr{\rho - \rho_0}{\rho_0}.
$$
Тогда
\begin{equation}\label{eq5}
\rho = \rho_0 (1+s).
\end{equation}

Тогда уравнение \eqref{eq3} перепишем в виде
\begin{equation}\label{eq6}
p=p_0(1+s)^\gamma.
\end{equation}
При малых колебаниях жидкости сжатие $s$ настолько мало, что высшими степенями  $s$ можно пренебречь. В результате из выражения \eqref{eq6} получим
\begin{equation}\label{eq7}
p=p_0(1+\gamma \cdot s).
\end{equation}

Подставим выражение \eqref{eq5} в уравнение неразрывности. Так как 
$$
\diver(\rho \bar{v})= \rho\diver(\bar{v})+\bar{v} \grad \rho = \rho_0 \diver(\bar{v}) + \rho_0 s \diver(\bar{v}) + \bar{v} \grad \rho,
$$
причем последними двумя слагаемыми можно пренебречь, то вместо уравнения \eqref{eq2} будем иметь
\begin{equation}\label{eq8}
\fr{\de s}{\de t} + \diver (\bar{v}) = 0.
\end{equation}
Уравнение \eqref{eq4} в том же приближении сводится к уравнению
\begin{equation}\label{eq9}
\fr{\de \bar{v}}{\de t} = -c^2\cdot\grad s,
\end{equation}
где $c=(\gamma \fr{p_0}{\rho_0})^{1/2}$ --- скорость звука.

Предположим теперь, что в начальный момент существует потенциал скоростей $\tilde{\psi_0}$, т.е.
\begin{equation}\label{eq 10}
\bar{v}\big|_{t=0}=\grad\tilde{\psi_0}.
\end{equation}

Из уравнения \eqref{eq9} имеем
$$
\bar{v}=\bar{v}|_{t=0}-c^2\grad \bigg(\int\limits_{0}^{t} s\diff t\bigg).
$$
С учетом \eqref{eq 10} получаем
\begin{equation}\label{eq 11}
\bar{v}=\grad\bigg[\tilde{\psi_0}-c^2\int\limits_{0}^{t} s\diff t\bigg]=\grad\tilde{\psi}.
\end{equation}
Это означает, что существует потенциал скоростей $\tilde{\psi}$ в любой момент времени $t:$
$$
\tilde{\psi}=\tilde{\psi_0}-c^2\int\limits_{0}^{t} s\diff t.
$$

Дифференцируя последнее выражение два раза по $t,$ получим
\begin{equation}\label{eq 12}
\fr{\de^2\tilde{\psi}}{\de t^2}=-c^2\fr{\de s}{\de t}.
\end{equation}

С другой стороны, подставляя выражение \eqref{eq 11} в уравнение \eqref{eq8}, будем иметь
\begin{equation}\label{eq 13}
\fr{\de s}{\de t}=-\diver\grad\tilde{\psi}=-\Lap\tilde{\psi}.
\end{equation}
Из уравнений \eqref{eq 12} и \eqref{eq 13} приходим к волновому уравнению
\begin{equation}\label{eq 14}
\fr{\de^2\tilde{\psi}}{\de t^2}=c^2\Lap\tilde{\psi},
\end{equation}
которое описывает процесс распространения звука в идеальной жидкости.

Отметим, что знания потенциала $\tilde{\psi}$ достаточно для определения всего процесса движения жидкости в случае малых возмущений, так как
$$
\bar{v}=\grad\tilde{\psi}\text{;}\quad s=-\fr{1}{c^2}\fr{\de\tilde{\psi}}{\de t}.
$$

Найдем акустическое давление $p'=p-p_0$. Из уравнения \eqref{eq4}, используя приближение, сделанное для \eqref{eq9}:
$$
\fr{\de \grad \tilde\psi}{\de t} = - \fr{1}{\rho_0}\grad p'.
$$

Занесем дифференциал под градиент и перенесем в левую часть:
$$
\grad \left(\fr{\de \tilde\psi}{\de t} + \fr{p'}{\rho_0}\right) = 0.
$$
Выражение в скобках не зависит от координат. Учитывая, что потенциал $\tilde\psi$ определяется с точностью до функции времени, приравняем выражение в скобках нулю:
$$
\fr{\de \tilde\psi}{\de t} + \fr{p'}{\rho_0} = 0.
$$

В случае установившегося режима колебаний
\begin{equation}\label{eq 15}
\tilde{\psi}=\psi \e^{-i\omega t}
\end{equation}
уравнение \eqref{eq 14} переходит в уравнение Гельмгольца
\begin{equation}\label{eq 16}
\Lap{\psi}+k^2\psi=0,
\end{equation}
где $\omega$ --- круговая частота; $k=\fr{\omega}{c}$ --- волновое число.
При этом акустическое давление 
$$
p'=-\rho_0 \fr{\de \tilde\psi}{\de t}=-\rho_0 \cdot (-i \omega) \psi \e^{-i\omega t}=i\rho_0\omega\tilde{\psi}.
$$


\newpage
\subsubsection{Распространение малых возмущений в упругой среде}
Под идеально упругим телом понимают тело, которое под воздействием приложенных к нему сил деформируется и полностью восстанавливает свою форму после устранения причины, вызвавшей деформацию. В качестве рассматриваемой среды возьмем модель линейной неоднородной изотропной упругой среды.

Уравнения движения сплошной среды, для случая отсутствия массовых сил, в криволинейной ортогональной системе координат имеют вид~\cite{Nowacki}:
\begin{equation}\label{eq moving}
\begin{split}
\fr{1}{h_1h_2h_3}\biggl[\fr{\de}{\de q_1} \biggl(h_2h_3\s_{11}\biggr) + \fr{\de}{\de q_2} \biggl(h_1h_3\s_{12}\biggr) +\fr{\de}{\de q_3} \biggl(h_1h_2\s_{13}\biggr) -\\
 -\s_{22}h_3\fr{\de h_2}{\de q_1} -\s_{33}h_2\fr{\de h_3}{\de q_1} + \s_{12}h_3\fr{\de h_1}{\de q_2} + \s_{13}h_2\fr{\de h_1}{\de q_3}\biggr] = \rho \fr{\de^2 u_1}{\de t^2};\\
 \fr{1}{h_1h_2h_3}\biggl[\fr{\de}{\de q_2} \biggl(h_1h_3\s_{22}\biggr) + \fr{\de}{\de q_3} \biggl(h_1h_2\s_{23}\biggr) +\fr{\de}{\de q_1} \biggl(h_2h_3\s_{12}\biggr) -\\
 -\s_{33}h_1\fr{\de h_3}{\de q_2} -\s_{11}h_3\fr{\de h_1}{\de q_2} + \s_{23}h_1\fr{\de h_2}{\de q_3} + \s_{12}h_3\fr{\de h_2}{\de q_1}\biggr] = \rho \fr{\de^2 u_2}{\de t^2};\\
 \fr{1}{h_1h_2h_3}\biggl[\fr{\de}{\de q_3} \biggl(h_1h_2\s_{33}\biggr) + \fr{\de}{\de q_1} \biggl(h_2h_3\s_{13}\biggr) +\fr{\de}{\de q_2} \biggl(h_1h_3\s_{23}\biggr) -\\
 -\s_{11}h_2\fr{\de h_1}{\de q_3} -\s_{22}h_1\fr{\de h_2}{\de q_3} + \s_{13}h_2\fr{\de h_3}{\de q_1} + \s_{23}h_1\fr{\de h_3}{\de q_2}\biggr] = \rho \fr{\de^2 u_3}{\de t^2}.\
\end{split}
\end{equation}
Здесь $\rho = \rho(\textbf{r})$ --- равновесная плотность среды ($\textbf{r}$ --- радиус-вектор точки тела);\\
$u_i$ --- компоненты вектора смещений $u;$\\
$\s_{ij}$ --- компоненты тензора напряжений;\\
$h_1,h_2,h_3$ --- коэффциенты Ламе криволинейной системы координат.

Обобщенный закон Гука --- связывает тензоры напряжения и деформации с помощью тензора упругих постоянных~\cite{Nowacki}:
\begin{equation}\label{Hook}
\s_{q_iq_j}=C_{ijkl}\eps_{q_kq_l}.
\end{equation}

В силу симметрии тензоров напряжении и деформации следующие компоненты тензора $C$ будут равны:
\begin{equation*}
\begin{split}
C_{ijkl}&=C_{ijlk};\\
C_{ijkl}&=C_{jilk};\\
C_{ijkl}&=C_{klij}.\\
\end{split}
\end{equation*}

Для случая неоднородного изотропного упругого тела модули упругости могут быть выражены через два независимых модуля упругости Ламе $\la, \mu:$
$$
C_{ijkl} = \mu(\delta_{ik}\delta_{jl}+\delta_{il}\delta_{jk}) + \la\delta_{ij}\delta{kl}.
$$
Подставляя в уравнение \eqref{Hook}, получим:
\begin{equation}\label{Hook2}
\s_{ij}=2\mu\eps_{ij}+\la\delta_{ij}\eps_{kk}.
\end{equation}

Компоненты тензора деформаций связаны с компонентами вектора смещения в ортогональной криволинейной системе координат $q_1,q_2,q_3$ следующими соотношениями~\cite{Nowacki}:
\begin{equation}\label{eq deform}
\begin{split}
\eps_{11} &= \fr{1}{h_1}\fr{\de u_1}{\de q_1} + \fr{1}{h_1h_2}\fr{\de h_1}{\de q_2}u_2 + \fr{1}{h_1h_3}\fr{\de h_1}{\de q_3}u_3;\\
\eps_{22} &= \fr{1}{h_2}\fr{\de u_2}{\de q_2} + \fr{1}{h_2h_3}\fr{\de h_2}{\de q_3}u_3 + \fr{1}{h_1h_2}\fr{\de h_2}{\de q_1}u_1;\\
\eps_{33} &= \fr{1}{h_3}\fr{\de u_3}{\de q_3} + \fr{1}{h_1h_3}\fr{\de h_3}{\de q_1}u_1 + \fr{1}{h_2h_3}\fr{\de h_3}{\de q_2}u_2;\\
\eps_{12} &= \fr12\left[\fr{h_1}{h_2}\fr{\de}{\de q_2}\left(\fr{u_1}{h_1}\right) + \fr{h_2}{h_1}\fr{\de}{\de q_1}\left(\fr{u_2}{h_2}\right) \right];\\
\eps_{13} &= \fr12\left[\fr{h_1}{h_3}\fr{\de}{\de q_3}\left(\fr{u_1}{h_1}\right) + \fr{h_3}{h_1}\fr{\de}{\de q_1}\left(\fr{u_3}{h_3}\right) \right];\\
\eps_{23} &= \fr12\left[\fr{h_2}{h_3}\fr{\de}{\de q_3}\left(\fr{u_2}{h_2}\right) + \fr{h_3}{h_2}\fr{\de}{\de q_2}\left(\fr{u_3}{h_3}\right) \right].\\
\end{split}
\end{equation}

Таким образом, математическая модель, описывающая распространение малых возмущений в изотропной неоднородной упругой среде состоит из уравнений движения сплошной среды \eqref{eq moving}, закона Гука \eqref{Hook2}, выражения компонентов тензора деформаций через компоненты вектора смещения \eqref{eq deform}, начальных и граничных условий.

\newpage
\section{ДИФРАКЦИЯ ЦИЛИНДРИЧЕСКИХ ЗВУКОВЫХ ВОЛН НА УПРУГОЙ СФЕРЕ С НЕОДНОРОДНЫМ ПОКРЫТИЕМ}
\subsection{Постановка задачи}

Пусть на однородную изотропную упругую сферу радиуса $r_2,$ находящуюся в идеальной сжимаемой жидкости и имеющую однородное покрытие с внешним радиусом $r_1,$ падает цилиндрическая звуковая волна, излучаемая бесконечно длинным линейным источником, с временной зависимостью $e^{-i\om t},$ где $\om$~--- круговая частота (в дальнейшем временной множитель будем опускать). 

Упругая однородная сфера, материал которой характеризуется равновесной плотностью $\rho_2$ и упругими коэффициентами Ламе $\la_2$ и $\mu_2,$ имеет покрытие в виде однородного изотропного упругого слоя. Модули упругости  материала однородного сферического слоя задаются как $\la_1$ и $\mu_1$ соответственно, а равновесная плотность слоя задается как $\rho_1.$ Окружающая сферический рассеиватель жидкость имеет в невозмущенном состоянии плотность $\rho_0$ и скорость звука $c.$

\begin{figure}[h]
\begin{center}
\begin{minipage}[h]{0.47\linewidth}
\includegraphics[width=1\linewidth]{sphere_layer2.eps}
\caption{<<Сфера со сферическим слоем>>}
\end{minipage}
%\hfill
%\begin{minipage}[h]{0.47\linewidth}
%\includegraphics[width=1\linewidth]{2a}
%\caption{<<при $ka=3.2$>>}
%\end{minipage}
\end{center}
\end{figure}

Требуется определить акустическое поле, рассеянное телом.


\newpage
\subsection{Аналитическое решение задачи}

Для решения задачи выберем сферическую $r, \Q, \f$ и цилиндрическую $\hat{r}, \f, z$ системы координат, начала которых совмещены с центром сферы. Линейный источник параллелен оси $z$ и имеет цилиндрические координаты $(\hat{r}_0, \f_0).$

Потенциал скоростей падающей волны имеет вид
$$
\Psi_0 = A_0H_0(kR),
$$
где $A_0$ --- амплитуда падающей волны;\\
$H_0(x)$ --- цилиндрическая функция Ханкеля первого рода нулевого порядка;\\
$R = \sqrt{\hat{r}^2+\hat{r}_0^2-2\hat{r}\hat{r}_0\cos\bigl(\f-\f_0\bigr)}$ --- расстояние от источника до произвольной точки внешнего пространства.

Движение однородной упругой среды в установившемся режиме описывается скалярным и векторным уравнениями Гельмгольца~\cite{Shenderov}
\begin{align}
\Lap \P &+ k_l^2 \P = 0;\label{helm1}\\
\Lap \F &+ k_\t^2\F = 0,\label{helm2}
\end{align}
где $\P$ и $\F$ --- скалярный и векторный потенциалы смещения; \\
$k_l = \om/c_l$ и $k_\t = \om/c_\t$ --- волновые числа продольных и поперечных упругих волн;\\
$c_l = \sqrt{\bigl(\la_2+2\mu_2\bigr)/\rho_2}$ и $c_\t = \sqrt{\bigl.\mu_2\bigr./\rho_2}$ --- скорости продольных и поперечных волн соответственно. При этом вектор смещения однородной среды $\mathbf{u}^0=\grad\P+\rot\F \q(\diver\F=0).$

Распространение упругих волн в неоднородном слое описывается общими уравнениями движения упругой среды, которые для установившегося режима движения в сферической системе координат имеют следующий вид~\cite{Nowacki}:
\begin{align}
\fr{\de\s_{rr}}{\de r} + \fr1r \fr{\de\s_{r\Q}}{\de\Q} + \fr{1}{r\sin\Q}\fr{\de\s_{r\f}}{\de\f} &+ \fr1r\biggl(2\s_{rr}-\s_{\Q\Q}-\s_{\f\f}+\s_{r\Q}\ctg\Q\biggr)=-\rho_1\om^2u_r;\notag\\
\fr{\de\s_{r\Q}}{\de r} + \fr1r \fr{\de\s_{\Q\Q}}{\de\Q} + \fr{1}{r\sin\Q}\fr{\de\s_{\Q\f}}{\de\f} &+ \fr1r\biggl(\left(\s_{\Q\Q}-\s_{\f\f}\right)\ctg\Q+3\s_{r\Q}\biggr)\!=\!-\rho_1\om^2u_\Q;\label{eq_moving3}\\
\fr{\de\s_{r\f}}{\de r} + \fr1r \fr{\de\s_{\Q\f}}{\de\Q} + \fr{1}{r\sin\Q}\fr{\de\s_{\f\f}}{\de\f} &+ \fr1r\biggl(3\s_{r\f}+2\s_{\Q\f}\ctg\Q\biggr)=-\rho_1\om^2u_\f,\notag
\end{align}
где $u_r, u_\Q, u_\f$ --- компоненты вектора смещения;\\
$\mathbf{u}$ и $\s_{ij}$ --- компоненты тензора напряжений неоднородной среды в сферической системе координат.

Во внешней области движение идеальной жидкости в установившемся режиме колебаний описывается уравнением Гельмгольца~\cite{Shenderov}
\begin{equation}\label{helm4}
\Lap \P_1 + k^2\P_1 = 0,
\end{equation}
где $\P_1 = \P_0 + \P_s$ --- потенциал скоростей полного акустического поля,\\
$\P_0$ и $\P_s$ --- потенциалы скоростей падающей и рассеянной волн соответственно,\\
$k=\om/c$ --- волновое число жидкости. При этом скорость частиц жидкости $\mathbf{v}$ и акустическое давление $p$ во внешней среде определяются по формулам:
$$
\mathbf{v} = \grad \P_1,\q p=i\rho_0\om\P_1.
$$


Граничные условия на внешней поверхности слоя заключаются в равенстве нормальных скоростей частиц упругой среды и жидкости, равенстве на ней нормального напряжения и акустического давления, отсутствии касательных напряжений. На внутренней поверхности слоя при переходе через границу раздела упругих сред должны быть непрерывны составляющие вектора смещения частиц, а также нормальные и тангенциальные напряжения. Имеем
при $r=r_1:$
$$
-i\om u_r=v_r,\q \s_{rr}=-p,\q \s_{r\Q}=0,\q \s_{r\f}=0,
$$
при $r=r_2:$
$$
u_r=u_r^0,\q u_\Q=u_\Q^0,\q u_\f=u_\f^0,\q \s_{rr}=\s_{rr}^0,\q \s_{r\Q}=\s_{r\Q}^0,\q \s_{r\f}=\s_{r\f}^0,
$$
где нулевыми индексами обозначены компоненты вектора смещения и тензора напряжений однородной упругой среды.

Потенциал $\P_s$ должен удовлетворять условиям излучения на бесконечности, а потенциалы $\P$ и $\F$ --- условию ограниченности.

Используя связь компонентов тензора напряжений с компонентами тензора деформаций (обобщенный закон Гука), а также выражения компонентов тензора деформаций через компоненты вектора смещения~\cite{Nowacki}, получаем в сферической системе координат следующие соотношения:
\begin{equation}\label{tensor comp5}
    \begin{gathered}
    \begin{aligned}
        \s_{rr} &= \biggl(\la_1+2\mu_1\biggr)\fr{\de u_r}{\de r} + \fr{\la_1} r \biggl(2u_r+\fr{\de u_\Q}{\de\Q}+\ctg\Q\;u_\Q + \fr{1}{\sin\Q}\fr{\de u_\f}{\de\f}\biggr);\\
        \s_{\Q\Q} &= \la_1\fr{\de u_r}{\de r} + \fr{2(\la_1+\mu_1)}r u_r + \fr{\la_1+2\mu_1}r \fr{\de u_\Q}{\de\Q} + \fr{\la_1} r \biggl(\ctg\Q \;u_\Q + \fr{1}{\sin\Q}\fr{\de u_\f}{\de\f}\biggr);\\
        \s_{\f\f} &= \la_1\fr{\de u_r}{\de r} + \fr{2(\la_1+\mu_1)}r u_r + \fr{\la_1} r \fr{\de u_\Q}{\de\Q} + \fr{\la_1+2\mu_1}r\biggl(\ctg\Q \;u_\Q + \fr{1}{\sin\Q}\fr{\de u_\f}{\de\f}\biggr);\\
    \end{aligned}\\
    \begin{aligned}
        \s_{r\Q}=\mu_1&\left(\fr1r \fr{\de u_r}{\de\Q} - \fr{u_\Q}{r} + \fr{\de u_\Q}{\de r}\right);\\
        \s_{r\f}=\mu_1&\left(\fr1{r\sin\Q} \fr{\de u_r}{\de\f} - \fr{u_\f}{r} + \fr{\de u_\f}{\de r}\right);\\
        \s_{\Q\f}=\fr{\mu_1} r&\left(\fr1{\sin\Q} \fr{\de u_\Q}{\de\f} + \fr{\de u_\f}{\de\Q} - \ctg\Q \;u_\f \right).\\
    \end{aligned}
    \end{gathered}
\end{equation}

Соотношения \eqref{tensor comp5} справедливы как для однородной упругой среды, так и для неоднородного слоя. В первом случае в выражениях \eqref{tensor comp5} компоненты вектора смещения и тензора напряжений должны быть снабжены индексом $0$, а также величины $\la_1$ и $\mu_1$ следует заменить на $\la_2$ и $\mu_2$ соответственно.

Цилиндрическая волна может быть представлена разложением~\cite{Mors Feshbach}:
\begin{equation*}
    \P_0=A_0\sum_{m=0}^\infty(2-\delta_{0m})\cos m(\f-\f_0)\left\{
                                                             \begin{array}{ll}
                                                             	J_m(k\hat{r})H_m(k\hat{r}_0), & \hat{r}<\hat{r}_0,\\
																J_m(k\hat{r}_0)H_m(k\hat{r}), & \hat{r}>\hat{r}_0,\\
                                                             \end{array}
                                                           \right.
\end{equation*}
где $J_m(x)$ и $H_m(x)$ --- цилиндрические функции Бесселя и Ханкеля первого рода порядков $m;$\\
$\delta_{0m}$ — символ Кронекера.

Хотя падающее поле двумерно, рассматриваемая задача является трехмерной.

Потенциал скоростей рассеянной волны $\P_s$ является решением уравнения \eqref{helm4}. С учетом условий излучения на бесконечности его будем искать в виде
\begin{equation}\label{velocity potential6}
\P_s = \sum_{n=0}^\infty\sum_{m=0}^n A_{mn}h_n(kr)P_n^m(\cos\Q)\cos m(\f-\f_0),
\end{equation}
где $h_n(x)$ --- сферическая функция Ханкеля первого рода порядка $n;$\\
$P_n^m(\cos{\Q})$ — присоединенный многочлен Лежандра степени $n$ порядка
$m.$

Сферическую функцию Ханкеля можно получить из цилиндрической функции Ханкеля первого порядка~\cite{Korn}:
$$
h_n(x)=\sqrt{\fr{\pi}{2x}}\:H_{\fr12+n}(x).
$$

Присоединенный многочлен Лежандра выражается через многочлен Лежандра~\cite{Korn}:
$$
P_m^n(x)=\Bigl(1-x^2\Bigr)^{\fr m2}\fr{\diff^m}{\diff x^m}P_n(x),
$$
а многочлен Лежандра по обобщенной формуле Родрига вычисляется как~\cite{Korn}\!:
$$
P_n(x)=\fr{1}{2^nn!}\fr{\diff^n}{\diff x^n}\biggl(x^2-1\biggr)^n.
$$
Скалярный потенциал смещения $\P$ являющийся решением уравнения \eqref{helm1}, с учетом условия ограниченности будем искать в виде
\begin{equation}\label{scalar potential bias7}
\P = \sum_{n=0}^\infty\sum_{m=0}^n B_{mn}j_n(k_lr)P_n^m(\cos\Q)\cos m(\f-\f_0),
\end{equation}
где $j_n(x)$ --- сферическая функция Бесселя порядка $n.$

Сферическую функцию Бесселя можно получить из цилиндрической функции Бесселя первого порядка~\cite{Korn}:
$$
j_n(x)=\sqrt{\fr{\pi}{2x}}\:J_{\fr12+n}(x).
$$

Представим векторный потенциал смещения $\F$ через две скалярные
функции $U$ и $V$~\cite{Mors Feshbach}:
$$
\F = \rot\rot(rU\mathbf{e_r}) + k_\t\rot(rV\mathbf{e_r}),
$$
где $\mathbf{e_r}$ --- орт сферической координатной оси $\mathbf{r}.$

В результате вместо векторного уравнения \eqref{helm2} получим два скалярных уравнения Гельмгольца относительно введенных скалярных функций
\begin{equation*}
\begin{split}
&\Lap U + k_\t^2U=0;\\
&\Lap V + k_\t^2V =0.
\end{split}
\end{equation*}

Компоненты вектора $\mathbf{u}^0$ выражаются через функции $U$ и $V$ следующим образом:
\begin{equation*}
\begin{split}
u_r^0 &= \fr{\de\P}{\de r} + k_\t\left(\fr{\de^2}{\de r^2}\biggl(rV\biggr) + k_\t^2rV\right),\\
u_\Q^0 &= \fr1r\fr{\de\P}{\de \Q} + \fr{k_\t}{r}\left(\fr{k_\t r}{\sin\Q}\fr{\de U}{\de \f} + \fr{\de^2}{\de r \de\Q}\biggl(rV\biggr)\right),\\
u_\f^0 &= \fr1{r\sin\Q}\fr{\de\P}{\de \f} + \fr{k_\t}{r}\left(\fr{1}{\sin\Q}\fr{\de^2}{\de r\de\f} \biggl(rV\biggr) + k_\t r\fr{\de U}{\de\Q}\right),\\
\end{split}
\end{equation*}

Функции $U$ и $V$ с учетом условия ограниченности будем искать в виде:
\begin{align}
U(r, \Q, \f) &= \sum_{n=0}^\infty\sum_{m=0}^n C_{mn}j_n(k_\t r)P_n^m(\cos\Q)\sin m(\f-\f_0),\label{function U8}\\
V(r, \Q, \f) &= \sum_{n=0}^\infty\sum_{m=0}^n D_{mn}j_n(k_\t r)P_n^m(\cos\Q)\cos m(\f-\f_0),\label{function V9}
\end{align}

При этом в разложениях \cref{velocity potential6,scalar potential bias7,function U8,function V9} вид зависимостей от $\f$ определяется соображениями симметрии векторов скорости $\mathbf{v}$ и смещения $\mathbf{u}^0$ относительно плоскости $\f=\f_0, \f_0+\pi$ (компоненты $v_r, u^0_r , u^0_\Q $ симметричны, а компонента $u^0_\f$ антисимметрична).

Теперь обратимся к уравнениям движения неоднородного слоя \eqref{eq_moving3}. Подставим выражения \eqref{tensor comp5} в \eqref{eq_moving3}. При этом введем новые функции $u_2$ и $u_3,$ связанные с $u_\Q$ и $u_\f$ соотношениями:
$$
u_\Q=\fr{\de u_2}{\de \Q}+ \fr{1}{\sin \Q}\fr{\de u_3}{\de \f};\q u_\f=\fr{1}{\sin \Q}\fr{\de u_2}{\de\f}-\fr{\de u_3}{\de \Q}.
$$
В результате приходим к системе уравнений, записанных относительно функций $u_r, u_2$ и $u_3:$
\begin{align}
	\begin{split}
		&\biggl(\la_1+2\mu_1\biggr)\fr{\de^2 u_r}{\de r^2}+\left(\la_1'+2\mu_1'+\fr{2(\la_1+2\mu_1)}{r}\right)\fr{\de u_r}{\de r}+\fr{\mu_1}{r^2}\opL(u_r)+\\
		+&\left[\fr2r \left(\la_1'-\fr{\la_1+2\mu_1}{r^2}\right)+\rho\om^2\right]u_r+\\ +\fr1r&\left[\biggl(\la_1+\mu_1\biggr)\fr{\de}{\de r}+\la_1'-\fr{\la_1+3\mu_1}{r}\right]\opL(u_2)=0,\\
	\end{split}\label{eq_10}\\
	\begin{split}
		\fr1r&\left[\biggl(\la_1+\mu_1\biggr)\fr{\de}{\de r}+\mu_1'+\fr{2(\la_1+2\mu_1)}{r}\right]\fr{\de u_r}{\de\Q}+\fr{\la_1+2\mu_1}{r^2} \fr{\de}{\de \Q}\opL(u_2)+\\
		+&\left[\mu_1\fr{\de^2}{\de r^2}+\left(\mu_1'+\fr{2\mu_1}{r}\right)\fr{\de}{\de r} -\fr{\mu_1'}{r}+\rho\om^2\right]\left(\fr{\de u_2}{\de \Q}+\fr{1}{\sin \Q}\fr{\de u_3}{\de\f}\right)+\\
		+&\fr{\mu_1}{r^2\sin\Q}\fr{\de}{\de\f}\opL(u_3)=0,
	\end{split}\label{eq_11}\\
	\begin{split}
		\fr1{r\sin\Q}&\left[\biggl(\la_1+\mu_1\biggr)\fr{\de}{\de r}+\mu_1'+\fr{2(\la_1+2\mu_1)}{r}\right]\fr{\de u_r}{\de\f}+\\
		+&\left[\mu_1\fr{\de^2}{\de r^2}+\left(\mu_1'+\fr{2\mu_1}{r}\right)\fr{\de}{\de r} -\fr{\mu_1'}{r}+\rho\om^2\right]\left(\fr{1}{\sin \Q}\fr{\de u_2}{\de \f}-\fr{\de u_3}{\de\Q}\right)+\\
		+&\fr{\la_1+2\mu_1}{r^2\sin\Q}\fr{\de}{\de\f}\opL(u_2)-\fr{\mu_1}{r^2}\fr{\de}{\de\Q}\opL(u_3)=0,
	\end{split}\label{eq_12}
\end{align}
где $\opL(\;)=\dfrac{\de^2}{\de\Q^2}+\ctg\Q\dfrac{\de}{\de\Q}+\dfrac{1}{\sin^2\Q}\dfrac{\de^2}{\de \f^2}.$

Проделаем следующие преобразования. Уравнение \eqref{eq_11} домножим на $\sin\Q$ и продифференцируем по $\Q,$ а уравнение \eqref{eq_12} продифференцируем по $\f.$ Складывая полученные уравнения, приходим к уравнению, содержащему только функции $u_r$ и $u_2:$
\begin{equation}\label{eq_13}
	\begin{split}
		\fr1r&\left[\biggl(\la_1+\mu_1\biggr)\fr{\de}{\de r}+\mu_1'+\fr{2(\la_1+2\mu_1)}{r}\right]\opL(u_r)+\\
		+&\left[\mu_1\fr{\de^2}{\de r^2}+\left(\mu_1'+\fr{2\mu_1}{r}\right)\fr{\de}{\de r}-\fr{\mu_1'}{r}+\rho\om^2+\fr{\la_1+2\mu_1}{r^2}\opL(\;)\right]\opL(u_2)=0.
	\end{split}
\end{equation}

Затем продифференцируем уравнение \eqref{eq_11} по $\f$ и вычтем уравнение \eqref{eq_12}, предварительно умноженное на $\sin\Q$ и продифференцированное по $\Q.$ Получим уравнение, в котором присутствует только функция $u_3:$
\begin{equation}\label{eq_14}
\left[\mu_1\fr{\de^2}{\de r^2}+\left(\mu_1'+\fr{2\mu_1}{r}\right)\fr{\de}{\de r}-\fr{\mu_1'}{r}+\rho\om^2+\fr{\mu_1}{r^2}\opL(\;)\right]\opL(u_3)=0.
\end{equation}
В результате приходим к системе, состоящей из уравнений \cref{eq_10,eq_13,eq_14}.

Функции $u_r, u_2, u_3$ будем искать в виде
\begin{equation}\label{func_u1u2u3_15}
	\begin{split}
		u_r(r, \Q, \f) &= \sum_{n=0}^\infty\sum_{m=0}^n U_{1mn}P_n^m(\cos\Q)\cos m(\f-\f_0),\\
		u_2(r, \Q, \f) &= \sum_{n=0}^\infty\sum_{m=0}^n U_{2mn}P_n^m(\cos\Q)\cos m(\f-\f_0),\\
		u_3(r, \Q, \f) &= \sum_{n=0}^\infty\sum_{m=0}^n U_{3mn}P_n^m(\cos\Q)\cos m(\f-\f_0),\\
	\end{split}
\end{equation}

При этом вид зависимостей от $\f$ в этих разложениях определяется соображениями симметрии вектора смещения $\mathbf{u}$ относительно плоскости $\f=\f_0, \f_0+\pi.$

Подставим разложения \eqref{func_u1u2u3_15} в уравнения \cref{eq_10,eq_13,eq_14}. Воспользовавшись уравнением для присоединенных многочленов Лежандра~\cite{Lebedev} и свойством ортогональности сферических гармоник~\cite{Lebedev}, получим для каждой пары индексов $m, n (n = 0, 1,\ldots;m \leq n)$ систему линейных однородных обыкновенных дифференциальных уравнений второго порядка относительно неизвестных функций $U_{1mn}(r), U_{2mn}(r) \text{ и } U_{3mn}(r):$
\begin{equation}\label{eq_16}
\hat A \hU''+\hat B\hU'+\hat C\hU=0,
\end{equation}
где $\hU=(U_{1mn},U_{2mn},U_{3mn})^{\T};$
$$
\hat A=
\left(\begin{array}{ccc}
	a_{11}&0&0\\
	0&a_{22}&0\\
	0&0&a_{33}
\end{array}\right), \q
\hat B=
\left(\begin{array}{ccc}
	b_{11}&b_{12}&0\\
	b_{21}&b_{22}&0\\
	0&0&b_{33}
\end{array}\right), \q
\hat C=
\left(\begin{array}{ccc}
	c_{11}&c_{12}&0\\
	c_{21}&c_{22}&0\\
	0&0&c_{33}
\end{array}\right),
$$
где
\begin{align*}
&a_{11}=\la_1+2\mu_1; &a_{22}&=a_{33}=\mu_1;\\
&b_{11}=\la_1'+2\mu_1'+\fr{2(\la_1+2\mu_1)}{r}; &b_{12}&=-\fr{n(n+1)(\la_1+\mu_1)}{r};\\
&b_{21}=\fr{\la_1+\mu_1}{r}; &b_{22}&=b_{33}=\fr{r\mu_1'+2\mu_1}{r};\\
&c_{11}=\rho\om^2+\fr{2r\la_1'-2(\la_1+2\mu_1)-n(n+1)\mu_1}{r^2};\\ 
&c_{12}=\fr{n(n+1)(\la_1+3\mu_1-r\la_1')}{r^2}; &c_{21}&=\fr{r\mu_1'+2(\la_1+2\mu_1)}{r^2};\\ &c_{22}=\rho\om^2-\fr{r\mu_1'+n(n+1)(\la_1+2\mu_1)}{r^2}; &c_{33}&=\rho\om^2-\fr{r\mu_1'+n(n+1)\mu_1}{r^2}.
\end{align*}

Анализ системы дифференциальных уравнений с переменными коэффициентами \eqref{eq_16} показывает, что все коэффициенты системы не зависят от индекса $m.$ В третье уравнение системы \eqref{eq_16} входит только функция $U_{3mn},$ причем в первые два уравнения этой системы она не входит.

Подставим разложения для падающей волны, \cref{velocity potential6,scalar potential bias7,function U8,function V9,func_u1u2u3_15} в граничные условия.

Из условия равенства нормальных скоростей при $r = r_1$ найдем коэффициенты $A_{mn},$ выраженные через величины $U_{1mn}(r_1).$ Для этого заменим в разложении для цилиндрической волны цилиндрическую координату $\hat r$ ее выражением $r \sin \Q$ в сферических координатах, применим интегральное соотношение~\cite{Mors Feshbach}:
$$
\int\limits_0^\pi J_m'(x\sin\Q)P_n^m(\cos\Q)\sin^2\Q\diff\Q=2i^{n-m}j_n'(x)P_n^m(0)
$$
и воспользуемся условиями ортогональности сферических гармоник. Здесь и далее штрихи означают дифференцирование по аргументу.

Получаем
$$
A_{mn}=-\fr{(2n+1)(n-m)!\bigl(2\ga_mi^{n-m}P_n^m(0)kj_n'(kr_1)+i\om U_{1mn}(r_1)\bigr)}{2(n+m)!kh_n'(kr_1)},
$$
где $\ga_m=A_0(2-\delta_{0m})H_m(k\hat{r}_0).$

Из условий непрерывности составляющих вектора смещения при $r = r_2$ найдем коэффициенты $B_{mn}, C_{mn}$ и $D_{mn},$ выраженные через величины\\
$U_{1mn}(r_2), U_{2mn}(r_2), U_{3mn}(r_2).$

Из условия $u_r=u_r^0$ при $r = r_2$ приходим к уравнению
\begin{equation}\label{eq_17}
\al_{1n}B_{mn}+\al_{2n}D_{mn}=U_{1mn}(r_2),
\end{equation}
где $\al_{1n}=k_lj_n'(k_lr_2);\q \al_{2n}=k_\t^2\bigl(k_\t r_2j_n''(k_\t r_2)+2j_n'(k_\t r_2)+k_\t r_2j_n(k_\t r_2)\bigr).$

Из условий $u_\Q = u_\Q^0,\; u_\f=u_\f^0$ при $r = r_2$ получаем два уравнения. Преобразуем эти уравнения. Сначала первое из них умножим на $\sin\Q,$ продифференцируем по $\Q$ и сложим со вторым уравнением, умноженным на $m.$ Получаем уравнение
\begin{equation}\label{eq_18}
\al_{3n}D_{mn}+\al_{4n}D_{mn}=U_{2mn}(r_2),
\end{equation}
где $\al_{3n}=\dfrac{j_n(k_lr_2)}{r_2};\q \al_{4n}=k_\t\left(\dfrac{j_n(k_\t r_2)}{r_2}+k_\t j_n'(k_\t r_2)\right).$

Затем первое уравнение умножим на $m$ и сложим со вторым уравнением, предварительно умноженным на $\sin\Q$ и продифференцированным по $\Q.$ В результате находим
$$
C_{mn}=\fr{U_{3mn}(r_2)}{k_\t^2j_n(k_\t r_2)}.
$$
Из системы уравнений \cref{eq_17,eq_18} получаем
$$
B_{mn}=\be_{1n}U_{1mn}(r_2)+\be_{2n}U_{2mn}(r_2),\q D_{mn}=\be_{3n}U_{1mn}(r_2)+\be_{4n}U_{2mn}(r_2),
$$
где $
\be_{1n}=\dfrac{\al_{4n}}{\Delta};\q
\be_{2n}=-\dfrac{\al_{2n}}{\Delta};\q
\be_{3n}=-\dfrac{\al_{3n}}{\Delta};\q
\be_{4n}=\dfrac{\al_{1n}}{\Delta};\q
\Delta=\al_{1n}\al_{4n}-\al_{2n}\al_{3n}.$

Из оставшихся неиспользованными граничных условий получим шесть краевых условий, которым должно удовлетворять решение системы дифференциальных уравнений \eqref{eq_16}. При этом будем использовать уравнение и условия ортогональности для присоединенных полиномов Лежандра, выражение для вронскиана~\cite{Lebedev} $j_n(x)h_n'(x)-j_n'(x)h_n(x)=i/x^2$ и интегральное соотношение~\cite{Mors Feshbach}
$$
\int\limits_0^\pi J_m(x\sin \Q)P_n^m(\cos \Q)\sin\Q\diff\Q = 2i^{n-m}j_n(x)P_n^m(0).
$$
Кроме того, при выводе краевых условий из последних двух граничных условий для $r = r_1$ и последних двух граничных условий для $r = r_2$ будем выполнять преобразования уравнений, аналогичные проведенным при получении уравнений \cref{eq_17,eq_18}.

В результате с учетом выражений для коэффициентов $B_{mn}, C_{mn}$ и $D_{mn}$ получаем следующие краевые условия:
\begin{equation}\label{boundary conditions_19}
	\begin{split}
		&\left(\hat A\hU'+\hat E\hU\right)_{r=r_1}=\hat D;\\
		&\left(\hat A\hU'+\hat F\hU\right)_{r=r_2}=0,\\
	\end{split}
\end{equation}
где
\begin{equation*}
	\begin{split}
		\hat E = &\left(\begin{array}{ccc}
			\dfrac{2\la_1}{r}+\dfrac{\om^2\rho_1h_n(kr)}{kh_n'(kr)}&-\dfrac{\la_1 n(n+1)}{r}&0\\
			\dfrac{\mu_1} r&-\dfrac{\mu_1} r&0\\
			0&0&-\dfrac{\mu_1} r
			\end{array}\right);\\
		\hat D=&\left(\begin{array}{ccc}
			\dfrac{2i^{n-m}P_n^m(0)\ga_m\om\rho_1}{(kr_1)^2h_n'(kr_1)},&0,&0
			\end{array}\right)^{\T};\q
		\hat F=\left(\begin{array}{ccc}
			f_{11}&f_{12}&0\\
			f_{21}&f_{22}&0\\
			0&0&f_{33}
		\end{array}\right),
	\end{split}
\end{equation*}
где
\begin{equation*}
	\begin{split}
		&\begin{aligned}
			f_{11}&=\fr{2\la_1}{r}-\be_{1n}g_{1n}-\be_{3n}g_{2n};& f_{12}&=-\fr{\la_1 n(n+1)}	 {r}-\be_{2n}g_{1n}-\be_{4n}g_{2n};\\
			f_{21}&=\fr{\mu_1} r -\mu_2(\be_{1n}g_{3n}+\be_{3n}g_{4n});& f_{22}&=\fr{\mu_1} r-\mu_2(\be_{2n}g_{3n}+\be_{4n}g_{4n});\\
			f_{33}&=-\fr{\mu_1} r -\fr{\mu_2g_{5n}}{j_n(k_\t r)};\\
		\end{aligned}\\
		&\begin{split}
			 g_{1n}&=(\la_2+2\mu_2)k_l^2j_n''(k_lr)+\fr{2\la_2k_lj_n'(k_lr)}{r}-\fr{\la_2n(n+1)j_n(k_lr)}{r^2};\\
			g_{2n}&=\la_2k_\t^3\bigl(k_\t rj_n'''(k_\t r)+5j_n''(k_\t r)+k_\t rj_n'(k_\t r)+3j_n(k_\t r)\bigr)+\\
			&+2\mu_2k_\t^3\bigl(k_\t rj_n'''(k_\t r)+3j_n''(k_\t r)+k_\t rj_n'(k_\t r)+j_n(k_\t r)\bigr)-\\
			&-\fr{\la_2n(n+1)k_\t j_n(k_\t r)}{r^2};\\
			g_{3n}&=\fr{2k_lrj_n'(k_lr)-2j_n(k_l r)}{r^2};\\
			g_{4n}&=k_\t\left(2k_\t^2j_n''(k_\t r)+\fr{2k_\t j_n'(k_\t r)}{r}+\left(k_\t^2-\fr{2}{r^2}\right)j_n(k_\t r)\right);\\
			g_{5n}&=k_\t j_n'(k_\t r) -\fr{j_n(k_\t r)}{r}.
		\end{split}
	\end{split}
\end{equation*}

Функция $U_{3mn}(r)$ не связана с функциями $U_{1mn}(r)$ и $U_{2mn}(r)$ не только в уравнениях системы \eqref{eq_16}, но и в краевых условиях \eqref{boundary conditions_19}. Так как дифференциальное уравнение и краевых условия для нахождения функции $U_{3mn}(r)$ однородны, то можно утверждать, что $U_{3mn}(r) \equiv 0.$ Тогда $C_{mn}=0.$

В результате получаем, что $u_3(r, \Q, \f) \equiv 0$ и $U(r, \Q, \f) \equiv 0.$

Элементы всех матриц, присутствующих в краевых условиях \eqref{boundary conditions_19}, за исключением $\hat D,$ не зависят от индекса $m.$ Причем индекс $m$ входит в $\hat D$ в виде множителя $\ga_m.$

Введем новые неизвестные функции $U_{1n}(r)$ и $U_{2n}(r)$ по формулам:
\begin{equation}\label{eq_20}
U_{1mn}(r)=\ga_mU_{1n}(r),\q U_{2mn}(r)=\ga_mU_{2n}(r)
\end{equation}

Тогда для нахождения функций $U_{1n}(r)$ и $U_{2n}(r)$ следует решить краевую задачу вида \eqref{eq_16}, \eqref{boundary conditions_19}, где под вектором $\hU$
следует понимать вектор $(U_{1n}, U_{2n})^{\T},$ а в $\hat D$ нужно исключить множитель $\ga_m.$

Для каждого $n (n = 0, 1, 2, \ldots)$ эта краевая задача может быть решена различными методами, например, изложенными в~\cite{inhomogeneous thermoelastic bodies}. Затем согласно \eqref{eq_20} находим функции $U_{1mn}(r)$ и $U_{2mn}(r)$ для всех $m = 0, 1, \ldots, n.$

После определения значений функций $U_{1n}(r)$ и $U_{2n}(r)$ при $r = r_1$ и $r =r_2$ вычисляем коэффициенты $A_{mn}, B_{mn}$ и $D_{mn}.$ В результате получаем аналитическое описание волновых полей вне тела, а также в упругом шаре и его покрытии.

Использование разложения \eqref{velocity potential6} при проведении численных расчетов позволяет исследовать влияние однородного покрытия на звукоотражающие характеристики упругого шара для различных материалов тела, волновых размеров.

Рассмотрим дальнюю зону акустического поля. Используя асимптотическое представление сферических функций Ханкеля при больших значениях аргумента $(kr \gg 1)$~\cite{Lebedev}
$$
h_n(kr)=(-i)^n\fr{\e^{ikr}}{kr},
$$
получаем выражение для потенциала скоростей рассеянной звуковой волны
$$
\P_s=\fr{r_1}{2r}\e^{ikr}F(\Q, \f),
$$
где
$$
F(\Q, \f)=\fr{2}{kr_1}\sum_{n=0}^\infty\sum_{m=0}^n (-i)^{n+1}A_{mn}P_n^m(\cos \Q)\cos m(\f-\f_0).
$$

\newpage
\subsection{Численное исследование и анализ результатов}
Были проведены расчеты амплитуды рассеяния $|F(\Q, \f)|$ для алюминиевого шара ($\rho_2 = 2.\!7 \cdot 10^3 \text{ кг/м}^3, \la_2 = 5.3 \cdot 10^{10} \text{ Н/м}^2, \mu_2 =2.6 \cdot 10^{10}\text{ Н/м}^2),$ находящегося в воде $(\rho_0 = 10^3 \text{ кг/м}^3, c = 1485 \text{ м/с}).$ Полагалось, что амплитуда падающей волны $A_0$ равна единице, а отношение внешнего радиуса $r1$ сферического слоя к внутреннему $r2$ равно $1.1$.

Расчеты проводились как для однородного покрытия с плотностью $\rho_1 = 1070 \text{ кг/м}^3$ и модулями упругости $\la_1 = 3.9 \cdot 10^9 \text{ Н/м}^2, \mu_1 = 9.8 \cdot 10^8 \text{ Н/м}^2$ (поливинилбутираль), так и для неоднородных покрытий, механические характеристики которых менялись по толщине сферического слоя по закону:
$$
\la=\la_1f(r), \q \mu=\mu_1f(r), \q \rho=\rho_0f(r).
$$

Были проанализированы два вида неоднородности материала покрытия:
$$
f_k(r)=a_k\left[\left(\fr{r-r_{3-k}}{r_2-r_1}\right)^2+1\right], \q k=1, 2.
$$

Множители $a_k$ выбраны так, чтобы средние по толщине слоя значения функций $f_k(r)$ были равны единице $(a_1=a_2=0.75)$. При этом в слое максимум функции $f_k(r)$ достигаеся при $r=r_k$ и минимум --- при $r=r_{3-k}.$

На рисунках \cref{pic1,pic2,pic3,pic4,pic5,pic6,pic7,pic8} представлены зависимости амплитуды рассеяния $|F(\Q, \f)|$ от полярного угла $\Q$ при волновых размерах тела $kr_1 = 4$ и $kr_1 = 7.$ Диаграммы направленности рассчитывались в плоскости $\f=0, \f=\pi.$ Полагалось $\f_0 = \pi.$ Расчеты проводились при различном удалении источника от препятствия: $k\hat r_0 = 8, 10, 15, 100.$

На полярных диаграммах направленности рассеянного поля сплошные линии соответствуют однородному покрытию, штриховые и штрих-пунктирные --- неоднородным покрытиям первого и второго видов соответственно. Направление распространения падающей волны --- слева направо --- показано стрелкой.

На рисунках \cref{pic9,pic10,pic11,pic12} представлены частотные характеристики --- зависимости $\left|F\left(\Q=\dfrac{\pi}{2}, \f=\pi\right)\right|$ от волнового размера шара $kr_1.$

\begin{figure}[!h]
\centering
\begin{minipage}[h]{0.4\linewidth}
\begin{minipage}[b][0.182\textheight][s]{0.05 \textwidth}
\centering
\vspace*{\stretch{1}}
$\rightarrow$
\vspace*{\stretch{1}}
\end{minipage}
  \includegraphics[width=0.85 \textwidth]{img4_8.eps}
  \caption{Амплитуда рассеяния для $kr_1=4, k\hat r_0=8$}\label{pic1}
\end{minipage}
\begin{minipage}[h]{0.4\linewidth}
\begin{minipage}[b][0.182\textheight][s]{0.05 \textwidth}
\centering
\vspace*{\stretch{1}}
$\rightarrow$
\vspace*{\stretch{1}}
\end{minipage}
  \includegraphics[width=0.85 \textwidth]{img7_8.eps}
  \caption{Амплитуда рассеяния для $kr_1=7, k\hat r_0=8$}\label{pic2}
\end{minipage}
\end{figure}

\begin{figure}[!h]
\begin{center}
\begin{minipage}[h]{0.4\linewidth}
\begin{minipage}[b][0.182\textheight][s]{0.05 \textwidth}
\centering
\vspace*{\stretch{1}}
$\rightarrow$
\vspace*{\stretch{1}}
\end{minipage}
  \includegraphics[width=0.85 \textwidth]{img4_10.eps}
  \caption{Амплитуда рассеяния для $kr_1=4, k\hat r_0=10$}\label{pic3}
\end{minipage}
\begin{minipage}[h]{0.4\linewidth}
\begin{minipage}[b][0.182\textheight][s]{0.05 \textwidth}
\centering
\vspace*{\stretch{1}}
$\rightarrow$
\vspace*{\stretch{1}}
\end{minipage}
  \includegraphics[width=0.85 \textwidth]{img7_10.eps}
  \caption{Амплитуда рассеяния для $kr_1=7, k\hat r_0=10$}\label{pic4}
\end{minipage}
\end{center}
\end{figure}

\begin{figure}[!h]
\begin{center}
\begin{minipage}[h]{0.4\linewidth}
\begin{minipage}[b][0.182\textheight][s]{0.05 \textwidth}
\centering
\vspace*{\stretch{1}}
$\rightarrow$
\vspace*{\stretch{1}}
\end{minipage}
  \includegraphics[width=0.85 \textwidth]{img4_15.eps}
  \caption{Амплитуда рассеяния для $kr_1=4, k\hat r_0=15$}\label{pic5}
\end{minipage}
\begin{minipage}[h]{0.4\linewidth}
\begin{minipage}[b][0.182\textheight][s]{0.05 \textwidth}
\centering
\vspace*{\stretch{1}}
$\rightarrow$
\vspace*{\stretch{1}}
\end{minipage}
  \includegraphics[width=0.85 \textwidth]{img7_15.eps}
  \caption{Амплитуда рассеяния для $kr_1=7, k\hat r_0=15$}\label{pic6}
\end{minipage}
\end{center}
\end{figure}

\begin{figure}[!h]
\begin{center}
\begin{minipage}[h]{0.4\linewidth}
\begin{minipage}[b][0.182\textheight][s]{0.05 \textwidth}
\centering
\vspace*{\stretch{1}}
$\rightarrow$
\vspace*{\stretch{1}}
\end{minipage}
  \includegraphics[width=0.85 \textwidth]{img4_100.eps}
  \caption{Амплитуда рассеяния для $kr_1=4, k\hat r_0=100$}\label{pic7}
\end{minipage}
\begin{minipage}[h]{0.4\linewidth}
\begin{minipage}[b][0.182\textheight][s]{0.05 \textwidth}
\centering
\vspace*{\stretch{1}}
$\rightarrow$
\vspace*{\stretch{1}}
\end{minipage}
  \includegraphics[width=0.85 \textwidth]{img7_100.eps}
  \caption{Амплитуда рассеяния для $kr_1=7, k\hat r_0=100$}\label{pic8}
\end{minipage}
\end{center}
\end{figure}

\begin{figure}[!h]
\begin{center}
\begin{minipage}[h]{0.45\linewidth}
  \includegraphics[width=0.9 \textwidth]{img_8.eps}
  \caption{Частотная характеристика для $k\hat r_0=8$}\label{pic9}
\end{minipage}
\begin{minipage}[h]{0.45\linewidth}
  \includegraphics[width=0.9 \textwidth]{img_10.eps}
  \caption{Частотная характеристика для $k\hat r_0=10$}\label{pic10}
\end{minipage}
\end{center}
\end{figure}

\begin{figure}[!h]
\begin{center}
\begin{minipage}[h]{0.45\linewidth}
  \includegraphics[width=0.9 \textwidth]{img_15.eps}
  \caption{Частотная характеристика для $k\hat r_0=15$}\label{pic11}
\end{minipage}
\begin{minipage}[h]{0.45\linewidth}
  \includegraphics[width=0.9 \textwidth]{img_100.eps}
  \caption{Частотная характеристика для $k\hat r_0=100$}\label{pic12}
\end{minipage}
\end{center}
\end{figure}

\newpage

\phantom{123}

\newpage

Анализируя частотные характеристики для различных видов покрытий, видим, что подбирая механические характеристики слоя, можно скорректировать зависимость амплитуды рассения от частоты падающей волны и добиться уменьшения размеров рассеяного поля для определенных частот. Кроме того, анализ графиков амплитуды рассеивания показывает влияние расстояния от источника падающей волны до рассеивателя на характеристики рассеивания.

\newpage

\section*{ЗАКЛЮЧЕНИЕ}
\addcontentsline{toc}{section}{ЗАКЛЮЧЕНИЕ}


В настоящей работе рассматривается задача дифракции цилиндрических звуковых волн на упругой сфере с неоднородным покрытием. 

 Построена математическая модель данной задачи.

 Получены аналитические решения задачи. 

На основе аналитического решения проведены численные исследования с помощью математического пакета Maple. Были получены зависимости амплитуды рассеивания цилиндрической звуковой волны и частотных характеристик от различных волновых размеров тела, различных видах неоднородности упругого слоя и различного удаления от источника волн.

\newpage

\input{bibl.tex}

\newpage
\section*{ПРИЛОЖЕНИЕ}
\addcontentsline{toc}{section}{ПРИЛОЖЕНИЕ}

Код для математического пакета Maple:
\begin{enumerate}
\item для вычисления диаграмм направленности:
\vspace{-0.5cm}
\begin{verbatim}
> restart; with(Physics); with(Physics[Vectors]); 
Setup(mathematicalnotation = true); 
with(Student[LinearAlgebra]); 
with(PDEtools); with(plots);
> At := Matrix(1 .. 3, 1 .. 3, shape = diagonal);
> At[1, 1] := proc (r) options operator, arrow; 
lambda(r)+2*mu(r) end proc; 
At[2, 2] := proc (r) options operator, arrow; mu(r) 
end proc; At[3, 3] := proc (r) options operator, arrow; mu(r) 
end proc;
> Bt := Matrix(1 .. 3, 1 .. 3);
> Bt[1, 1] := proc (r) options operator, arrow; 
diff(lambda(r), r)+2*(diff(mu(r), r))
+(2*lambda(r)+4*mu(r))/r end proc; 
Bt[1, 2] := proc (r) options operator, arrow; 
-n*(n+1)*(lambda(r)+mu(r))/r end proc; Bt[1, 3] := 0; 
Bt[2, 1] := proc (r) options 
operator, arrow; (lambda(r)+mu(r))/r end proc; 
Bt[2, 2] := proc (r) options operator, arrow; 
(r*(diff(mu(r), r))+2*mu(r))/r end proc; 
Bt[2, 3] := 0; Bt[3, 1] := 0; 
Bt[3, 2] := 0; Bt[3, 3] := proc (r) 
options operator, arrow; (r*(diff(mu(r), r))
+2*mu(r))/r end proc;
> Ct := Matrix(1 .. 3, 1 .. 3);
> Ct[1, 1] := proc (r) options operator, arrow; ro(r)*omega^2
+(2*r*(diff(lambda(r), r))
-2*lambda(r)-4*mu(r)-n*(n+1)*mu(r))/r^2 
end proc; Ct[1, 2] := proc (r) 
options operator, arrow; n*(n+1)
*(lambda(r)+3*mu(r)-r*(diff(lambda(r), r)))/r^2 end proc; 
Ct[1, 3] := 0; Ct[2, 1] := proc (r) options operator, arrow; 
(r*(diff(mu(r), r))+2*lambda(r)+4*mu(r))/r^2 end proc; 
Ct[2, 2] := proc (r) options operator, arrow; ro(r)*omega^2
-(r*(diff(mu(r), r))+n*(n+1)
*(lambda(r)+2*mu(r)))/r^2 end proc; 
Ct[2, 3] := 0; Ct[3, 1] := 0; 
Ct[3, 2] := 0; Ct[3, 3] := proc (r) 
options operator, arrow; ro(r)*omega^2-(r*(diff(mu(r), r))
+n*(n+1)*mu(r))/r^2 end proc;
> du16Re[1] := Re(At[1, 1](r))*(diff(U1nRe(r), r, r))
-Im(At[1, 1](r))*(diff(U1nIm(r), r, r))+Re(Bt[1, 1](r))
*(diff(U1nRe(r), r))-Im(Bt[1, 1](r))*(diff(U1nIm(r), r))
+Re(Bt[1, 2](r))*(diff(U2nRe(r), r))-Im(Bt[1, 2](r))
*(diff(U2nIm(r), r))+Re(Ct[1, 1](r))*U1nRe(r)-Im(Ct[1, 1](r))
*U1nIm(r)+Re(Ct[1, 2](r))*U2nRe(r)
-Im(Ct[1, 2](r))*U2nIm(r) = 0;
> du16Im[1] := Im(At[1, 1](r))*(diff(U1nRe(r), r, r))
+Re(At[1, 1](r))*(diff(U1nIm(r), r, r))+Im(Bt[1, 1](r))
*(diff(U1nRe(r), r))+Re(Bt[1, 1](r))*(diff(U1nIm(r), r))
+Im(Bt[1, 2](r))*(diff(U2nRe(r), r))+Re(Bt[1, 2](r))
*(diff(U2nIm(r), r))+Im(Ct[1, 1](r))*U1nRe(r)+Re(Ct[1, 1](r))
*U1nIm(r)+Im(Ct[1, 2](r))*U2nRe(r)
+Re(Ct[1, 2](r))*U2nIm(r) = 0;
> du16Re[2] := Re(At[2, 2](r))*(diff(U2nRe(r), r, r))
-Im(At[2, 2](r))*(diff(U2nIm(r), r, r))+Re(Bt[2, 1](r))
*(diff(U1nRe(r), r))-Im(Bt[2, 1](r))*(diff(U1nIm(r), r))
+Re(Bt[2, 2](r))*(diff(U2nRe(r), r))-Im(Bt[2, 2](r))
*(diff(U2nIm(r), r))+Re(Ct[2, 1](r))*U1nRe(r)-Im(Ct[2, 1](r))
*U1nIm(r)+Re(Ct[2, 2](r))*U2nRe(r)
-Im(Ct[2, 2](r))*U2nIm(r) = 0;
> du16Im[2] := Im(At[2, 2](r))*(diff(U2nRe(r), r, r))
+Re(At[2, 2](r))*(diff(U2nIm(r), r, r))+Im(Bt[2, 1](r))
*(diff(U1nRe(r), r))+Re(Bt[2, 1](r))*(diff(U1nIm(r), r))
+Im(Bt[2, 2](r))*(diff(U2nRe(r), r))+Re(Bt[2, 2](r))
*(diff(U2nIm(r), r))+Im(Ct[2, 1](r))*U1nRe(r)+Re(Ct[2, 1](r))
*U1nIm(r)+Im(Ct[2, 2](r))*U2nRe(r)
+Re(Ct[2, 2](r))*U2nIm(r) = 0;
> Legendre1 := proc (a, z) options operator, arrow; 
subs(x = z, diff((x^2-1)^a, [`$`(x, a)]))/(2^a*factorial(a)) 
end proc;
> Legendre2 := proc (a, b, z) options operator, arrow; 
(1-z^2)^((1/2)*b)*subs(x = z, diff(Legendre1(a, x), 
[`$`(x, b)])) end proc;
> for i from 0 to 5 do for j from 0 to 20 do Leg[i, j] := 
simplify(Legendre2(i, j, 0)) end do end do; Leg[0, 4];
> Dt[1] := proc (r) options 
operator, arrow; 2*I^(n-m)*Leg[n, m]
*omega*ro1/(k^2*r1^2*subs(x = k*r1, diff(sqrt((1/2)*Pi/x)
*HankelH1(n+1/2, x), x))) end proc; Dt[2] := 0; Dt[3] := 0;
> Et[1, 1] := proc (r) options operator, arrow;
 2*lambda(r)/r+omega^2*ro1*sqrt((1/2)*Pi/(k*r))
*HankelH1(n+1/2, k*r)/(k*subs(x = k*r, diff(sqrt((1/2)*Pi/x)
*HankelH1(n+1/2, x), x))) end proc; Et[1, 2] := proc (r) 
options operator, arrow; -lambda(r)*n*(n+1)/r end proc; 
Et[1, 3] := 0; Et[2, 1] := proc (r) options operator, arrow; 
mu(r)/r end proc; Et[2, 2] := proc (r) options operator, 
arrow; -mu(r)/r end proc; Et[2, 3] := 0; Et[3, 1] := 0; 
Et[3, 2] := 0; Et[3, 3] := proc (r) options operator, arrow;
 -mu(r)/r end proc;
>`&alpha;1n` := kl*subs(x = kl*r0, diff(sqrt(Pi/(2*x))
*BesselJ(n+1/2, x), x)); `&alpha;2n` := `k&tau;`^2
*(`k&tau;`*r0*subs(x = `k&tau;`*r0, diff(sqrt(Pi/(2*x))
*BesselJ(n+1/2, x), x, x))+2*subs(x = `k&tau;`*r0, 
diff(sqrt(Pi/(2*x))*BesselJ(n+1/2, x), x))+`k&tau;`*r0
*subs(x = `k&tau;`*r0, sqrt(Pi/(2*x))*BesselJ(n+1/2, x)));
 `&alpha;3n` := subs(x = kl*r0, sqrt(Pi/(2*x))
 *BesselJ(n+1/2, x))/r0; `&alpha;4n` := `k&tau;`
 *(subs(x = `k&tau;`*r0, sqrt(Pi/(2*x))*BesselJ(n+1/2, x))
 /r0+`k&tau;`*subs(x = `k&tau;`*r0, diff(sqrt(Pi/(2*x))
 *BesselJ(n+1/2, x), x)));
> `&alpha;delta` := `&alpha;1n`*`&alpha;4n`
-`&alpha;2n`*`&alpha;3n`;
>`&beta;1n` := `&alpha;4n`/`&alpha;delta`; 
`&beta;2n` := -`&alpha;2n`/`&alpha;delta`; 
`&beta;3n` := -`&alpha;3n`/`&alpha;delta`; 
`&beta;4n` := `&alpha;1n`/`&alpha;delta`;
> g1n := proc (r) options operator, arrow; (lambda0+2*mu0)
*kl^2*subs(x = kl*r, diff(sqrt((1/2)*Pi/x)
*BesselJ(n+1/2, x), x, x))+2*lambda0*kl*subs(x = kl*r,
 diff(sqrt((1/2)*Pi/x)*BesselJ(n+1/2, x), x))/r
-lambda0*n*(n+1)*subs(x = kl*r, sqrt((1/2)*Pi/x)
*BesselJ(n+1/2, x))/r^2 end proc;
> g2n := proc (r) options operator, arrow; lambda0
*`k&tau;`^3*(`k&tau;`*r*subs(x = `k&tau;`*r, 
diff(sqrt((1/2)*Pi/x)*BesselJ(n+1/2, x), x, x, x))
+5*subs(x = `k&tau;`*r, diff(sqrt((1/2)*Pi/x)
*BesselJ(n+1/2, x), x, x))+`k&tau;`*r*subs(x = `k&tau;`*r,
 diff(sqrt((1/2)*Pi/x)*BesselJ(n+1/2, x), x))
+3*subs(x = `k&tau;`*r, sqrt((1/2)*Pi/x)
*BesselJ(n+1/2, x)))+2*mu0*`k&tau;`^3*(`k&tau;`*r
*subs(x = `k&tau;`*r, diff(sqrt((1/2)*Pi/x)
*BesselJ(n+1/2, x), x, x, x))+3*subs(x = `k&tau;`*r, 
diff(sqrt((1/2)*Pi/x)*BesselJ(n+1/2, x), x, x))
+`k&tau;`*r*subs(x = `k&tau;`*r, diff(sqrt((1/2)*Pi/x)
*BesselJ(n+1/2, x), x))+subs(x = `k&tau;`*r, sqrt((1/2)*Pi/x)
*BesselJ(n+1/2, x)))-lambda0*n*(n+1)*`k&tau;`
*subs(x = `k&tau;`*r, sqrt((1/2)*Pi/x)
*BesselJ(n+1/2, x))/r end proc;
> g3n := proc (r) options operator, arrow; 
(2*kl*subs(x = kl*r, diff(sqrt((1/2)*Pi/x)
*BesselJ(n+1/2, x), x))-2*subs(x = kl*r, 
sqrt((1/2)*Pi/x)*BesselJ(n+1/2, x))/r)/r end proc;
> g4n := proc (r) options operator, arrow; `k&tau;`
*(2*`k&tau;`^2*subs(x = `k&tau;`*r, diff(sqrt((1/2)*Pi/x)
*BesselJ(n+1/2, x), x, x))+2*`k&tau;`
*subs(x = `k&tau;`*r, diff(sqrt((1/2)*Pi/x)
*BesselJ(n+1/2, x), x))/r+(`k&tau;`^2-2/r^2)
*subs(x = `k&tau;`*r, sqrt((1/2)*Pi/x)*BesselJ(n+1/2, x)))
 end proc;
> g5n := proc (r) options operator, arrow; `k&tau;`
*subs(x = `k&tau;`*r, diff(sqrt((1/2)*Pi/x)
*BesselJ(n+1/2, x), x))-subs(x = `k&tau;`*r, sqrt((1/2)*Pi/x)
*BesselJ(n+1/2, x))/r end proc;
> Ft[1, 1] := proc (r) options operator, arrow; 
2*lambda(r)/r-`&beta;1n`*g1n(r)-`&beta;3n`*g2n(r) end proc; 
Ft[1, 2] := proc (r) options operator, arrow; 
-lambda(r)*n*(n+1)/r-`&beta;2n`*g1n(r)-`&beta;4n`*g2n(r) 
end proc; Ft[1, 3] := 0; 
Ft[2, 1] := proc (r) options operator, 
arrow; mu(r)/r-mu0*(`&beta;1n`*g3n(r)+`&beta;3n`*g4n(r)) 
end proc; Ft[2, 2] := proc (r) options 
operator, arrow; mu(r)/r
-mu0*(`&beta;2n`*g3n(r)+`&beta;4n`*g4n(r)) end proc; 
Ft[2, 3] := 0; Ft[3, 1] := 0; Ft[3, 2] := 0;
Ft[3, 3] := proc (r) options operator, arrow; -mu(r)/r
-mu0*g5n(r)/subs(x = `k&tau;`*r, sqrt((1/2)*Pi/x)
*BesselJ(n+1/2, x)) end proc;
> gru1Re[1] := Re(At[1, 1](r1))*(D(U1nRe))(r1)
-Im(At[1, 1](r1))*(D(U1nIm))(r1)+Re(Et[1, 1](r1))
*U1nRe(r1)-Im(Et[1, 1](r1))*U1nIm(r1)+Re(Et[1, 2](r1))
*U2nRe(r1)-Im(Et[1, 2](r1))*U2nIm(r1)-Re(Dt[1](r1)) = 0;
> gru1Im[1] := Im(At[1, 1](r1))*(D(U1nRe))(r1)
+Re(At[1, 1](r1))*(D(U1nIm))(r1)+Im(Et[1, 1](r1))
*U1nRe(r1)+Re(Et[1, 1](r1))*U1nIm(r1)+Im(Et[1, 2](r1))
*U2nRe(r1)+Re(Et[1, 2](r1))*U2nIm(r1)-Im(Dt[1](r1)) = 0;
> gru1Re[2] := Re(At[2, 2](r1))*(D(U2nRe))(r1)
-Im(At[2, 2](r1))*(D(U2nIm))(r1)+Re(Et[2, 1](r1))
*U1nRe(r1)-Im(Et[2, 1](r1))*U1nIm(r1)+Re(Et[2, 2](r1))
*U2nRe(r1)-Im(Et[2, 2](r1))*U2nIm(r1)-Re(Dt[2](r1)) = 0;
> gru1Im[2] := Im(At[2, 2](r1))*(D(U2nRe))(r1)
+Re(At[2, 2](r1))*(D(U2nIm))(r1)+Im(Et[2, 1](r1))
*U1nRe(r1)+Re(Et[2, 1](r1))*U1nIm(r1)+Im(Et[2, 2](r1))
*U2nRe(r1)+Re(Et[2, 2](r1))*U2nIm(r1)-Im(Dt[2](r1)) = 0;
> gru2Re[1] := Re(At[1, 1](r0))*(D(U1nRe))(r0)
-Im(At[1, 1](r0))*(D(U1nIm))(r0)+Re(Ft[1, 1](r0))
*U1nRe(r0)-Im(Ft[1, 1](r0))*U1nIm(r0)+Re(Ft[1, 2](r0))
*U2nRe(r0)-Im(Ft[1, 2](r0))*U2nIm(r0) = 0;
> gru2Im[1] := Im(At[1, 1](r0))*(D(U1nRe))(r0)
+Re(At[1, 1](r0))*(D(U1nIm))(r0)+Im(Ft[1, 1](r0))
*U1nRe(r0)+Re(Ft[1, 1](r0))*U1nIm(r0)+Im(Ft[1, 2](r0))
*U2nRe(r0)+Re(Ft[1, 2](r0))*U2nIm(r0) = 0;
> gru2Re[2] := Re(At[2, 2](r0))*(D(U2nRe))(r0)
-Im(At[2, 2](r0))*(D(U2nIm))(r0)+Re(Ft[2, 1](r0))
*U1nRe(r0)-Im(Ft[2, 1](r0))*U1nIm(r0)+Re(Ft[2, 2](r0))
*U2nRe(r0)-Im(Ft[2, 2](r0))*U2nIm(r0) = 0;
> gru2Im[2] := Im(At[2, 2](r0))*(D(U2nRe))(r0)
+Re(At[2, 2](r0))*(D(U2nIm))(r0)+Im(Ft[2, 1](r0))
*U1nRe(r0)+Re(Ft[2, 1](r0))*U1nIm(r0)+Im(Ft[2, 2](r0))
*U2nRe(r0)+Re(Ft[2, 2](r0))*U2nIm(r0) = 0;
> `&gamma;m` := proc (m) options operator, arrow; 
A0*(2-KroneckerDelta[0, m])*HankelH1(m, k*rho0) end proc;
> A := proc (n, m) options operator, arrow; -(1/2)
*(2*n+1)*factorial(n-m)*(2*`&gamma;m`(m)*I^(n-m)
*Leg[n, m]*k*subs(r = k*r1, diff(sqrt((1/2)*Pi/r)
*BesselJ(n+1/2, r), r))+I*omega*subs(op(2, s[n, m](r1)), 
op(4, s[n, m](r1)), U1nRe(r)+I*U1nIm(r))*`&gamma;m`(m))
/(factorial(n+m)*k*subs(r = k*r1, diff(sqrt((1/2)*Pi/r)
*HankelH1(n+1/2, r), r))) end proc;
> F := proc (theta, phi) options operator, arrow; 
2*add(add((-I)^(n+1)*A(n, m)*LegendreP(n, m, cos(theta))
*cos(m*(phi-phi0)), m = 0 .. n), n = 0 .. 5)/(k*r1) end proc;
> cl := sqrt((lambda0+2*mu0)/ro0); `c&tau;` := sqrt(mu0/ro0); 
kl := omega/cl; `k&tau;` := omega/`c&tau;`;
> r0 := 10; r1 := Physics[`*`](1.1, r0); k := Physics[`*`](10, 
Physics[`^`](r1, -1)); ro0 := Physics[`*`](2.7, 
Physics[`^`](10, 3)); lambda0 := Physics[`*`](5.3, 
Physics[`^`](10, 10)); mu0 := Physics[`*`](2.6, 
Physics[`^`](10, 10)); ro := proc (r) 
options operator, arrow; 1070*fk(r) end proc; lambda := proc (r) 
options operator, arrow; 3.9*10^9*fk(r) end proc; mu := proc (r) 
options operator, arrow; 9.8*10^8*fk(r) end proc; ro1 := 
Physics[`^`](10, 3); c := 1485; A0 := 1; 
omega := Physics[`*`](k, c); 
rho0 := Physics[`*`](100, Physics[`^`](k, -1)); phi0 := Pi
> psi_s := proc (r, theta, phi) options operator, arrow; 
(1/2)*r1*exp(I*k*r)*F(theta, phi)/r end proc;
>fk := proc (r) options operator, arrow; 1 end proc; 
s := array(0 .. 10, 0 .. 10); for i from 0 to 10 do 
for j from 0 to 10 do s[i, j] := dsolve(eval(eval(
{du16Im[1], du16Im[2], du16Re[1], du16Re[2], gru1Im[1], 
gru1Im[2], gru1Re[1], gru1Re[2], gru2Im[1], gru2Im[2], 
gru2Re[1], gru2Re[2]}, n = i), m = j), [U1nRe(r), U1nIm(r), 
U2nRe(r), U2nIm(r)], numeric) end do end do; p1 := 
polarplot(abs(F(Physics[Vectors][`+`](Physics[`*`](Pi, 
Physics[`^`](2, -1)), -theta), 0)), theta = -Physics[`*`](Pi, 
Physics[`^`](2, -1)) .. Physics[`*`](Pi, Physics[`^`](2, -1)), 
color = black, linestyle = solid); p2 := polarplot(abs(
F(Physics[Vectors][`+`](theta, -Physics[`*`](Pi, 
Physics[`^`](2, -1))), Pi)), theta = Physics[`*`](Pi, 
Physics[`^`](2, -1)) .. Physics[`*`](Physics[`*`](3, Pi), 
Physics[`^`](2, -1)), color = black, linestyle = solid); 
p7 := display(p1, p2); fk := proc (r) options operator, arrow;
.75*((r-r1)^2/(r1-r0)^2+1) end proc; s := array(0 .. 10,
0 .. 10); for i from 0 to 10 do for j from 0 to 10 do
s[i, j] := dsolve(eval(eval({du16Im[1], du16Im[2], 
du16Re[1], du16Re[2], gru1Im[1], gru1Im[2], gru1Re[1], 
gru1Re[2], gru2Im[1], gru2Im[2], gru2Re[1], gru2Re[2]}, 
n = i), m = j), [U1nRe(r), U1nIm(r), U2nRe(r), U2nIm(r)], 
numeric) end do end do; p3 := polarplot(abs(F(
Physics[Vectors][`+`](Physics[`*`](Pi, Physics[`^`](2, -1)),
-theta), 0)), theta = -Physics[`*`](Pi, 
Physics[`^`](2, -1)) .. Physics[`*`](Pi, 
Physics[`^`](2, -1)), color = black, linestyle = dot); 
p4 := polarplot(abs(F(Physics[Vectors][`+`](theta, 
-Physics[`*`](Pi, Physics[`^`](2, -1))), Pi)), 
theta = Physics[`*`](Pi, Physics[`^`](2, -1))
.. Physics[`*`](Physics[`*`](3, Pi), Physics[`^`](2, -1)), 
color = black, linestyle = dot); p8 := display(p3, p4); 
fk := proc (r) options operator, arrow; .75*((r-r0)^2/
(r1-r0)^2+1) end proc; s := array(0 .. 10, 0 .. 10); 
for i from 0 to 10 do for j from 0 to 10 do 
s[i, j] := dsolve(eval(eval({du16Im[1], du16Im[2], 
du16Re[1], du16Re[2], gru1Im[1], gru1Im[2], 
gru1Re[1], gru1Re[2], gru2Im[1], gru2Im[2], 
gru2Re[1], gru2Re[2]}, n = i), m = j), [U1nRe(r), 
U1nIm(r), U2nRe(r), U2nIm(r)], numeric) end do 
end do; p5 := polarplot(abs(F(Physics[Vectors][`+`]
(Physics[`*`](Pi, Physics[`^`](2, -1)), -theta), 0)), 
theta = -Physics[`*`](Pi, Physics[`^`](2, -1)) ..
Physics[`*`](Pi, Physics[`^`](2, -1)), color = black, 
linestyle = dashdot); p6 := polarplot(abs(F(
Physics[Vectors][`+`](theta, -Physics[`*`](Pi, 
Physics[`^`](2, -1))), Pi)), theta = Physics[`*`](Pi, 
Physics[`^`](2, -1)) .. Physics[`*`](Physics[`*`](3, 
Pi), Physics[`^`](2, -1)), color = black, 
linestyle = dashdot); p9 := display(p5, p6):
> display(p7, p8, p9);
\end{verbatim}
\vspace{-0.5cm}
\item для вычисления графика частотных характеристик (дополнение к первой части):
\vspace{-0.5cm}
\begin{verbatim}
> omega := 'omega'; k := Physics[`*`](omega, 
Physics[`^`](c, -1)); r0 := 10; r1 := Physics[`*`](1.1, 
r0); ro0 := Physics[`*`](2.7, Physics[`^`](10, 3)); 
lambda0 := Physics[`*`](5.3, Physics[`^`](10, 10)); 
mu0 := Physics[`*`](2.6, Physics[`^`](10, 10)); 
ro1 := Physics[`^`](10, 3); c := 1485; A0 := 1; 
lambda := proc (r) options operator, arrow; 
3.9*10^9*fk(r) end proc; mu := proc (r) options 
operator, arrow; 9.8*10^8*fk(r) end proc; 
ro := proc (r) options operator, arrow; 1070*fk(r) 
end proc; rho0 := Physics[`*`](100, Physics[`^`](k, -1)); 
phi0 := Pi;
> fk := proc (r) options operator, arrow; 1 end proc; 
omega := .1; for i from 0 to 5 do for j from 0 to 5 do 
s[i, j] := dsolve(eval(eval({du16Im[1], du16Im[2], 
du16Re[1], du16Re[2], gru1Im[1], gru1Im[2], gru1Re[1], 
gru1Re[2], gru2Im[1], gru2Im[2], gru2Re[1], gru2Re[2]}, 
n = i), m = j), [U1nRe(r), U1nIm(r), U2nRe(r), U2nIm(r)], 
numeric, maxmesh = 256) end do end do; F_r1 := [Physics[`*`](k, 
r1), evalf(abs(F(Physics[`*`](Pi, Physics[`^`](2, -1)), 
Pi)))]; for omega by 50 to 10000 do for i from 0 to 5 do 
for j from 0 to 5 do s[i, j] := dsolve(eval(eval({du16Im[1], 
du16Im[2], du16Re[1], du16Re[2], gru1Im[1], gru1Im[2], 
gru1Re[1], gru1Re[2], gru2Im[1], gru2Im[2], gru2Re[1], 
gru2Re[2]}, n = i), m = j), [U1nRe(r), U1nIm(r), U2nRe(r), 
U2nIm(r)], numeric, maxmesh = 256) end do end do; 
F_r1 := F_r1, [Physics[`*`](k, r1), evalf(abs(F(
Physics[`*`](Pi, Physics[`^`](2, -1)), Pi)))] end do; 
p1 := plot([F_r1], color = black, linestyle = solid); 
fk := proc (r) options operator, arrow; .75*((r-r1)^2/
(r1-r0)^2+1) end proc; omega := .1; for i from 0 to 5 do 
for j from 0 to 5 do s[i, j] := dsolve(eval(eval({
du16Im[1], du16Im[2], du16Re[1], du16Re[2], gru1Im[1], 
gru1Im[2], gru1Re[1], gru1Re[2], gru2Im[1], gru2Im[2], 
gru2Re[1], gru2Re[2]}, n = i), m = j), [U1nRe(r), 
U1nIm(r), U2nRe(r), U2nIm(r)], numeric, maxmesh = 256) end 
do end do; F_r1 := [Physics[`*`](k, r1), evalf(abs(F(
Physics[`*`](Pi, Physics[`^`](2, -1)), Pi)))]; for omega 
by 50 to 10000 do for i from 0 to 5 do for j from 0 to 5 do 
s[i, j] := dsolve(eval(eval({du16Im[1], du16Im[2], 
du16Re[1], du16Re[2], gru1Im[1], gru1Im[2], gru1Re[1], 
gru1Re[2], gru2Im[1], gru2Im[2], gru2Re[1], gru2Re[2]}, 
n = i), m = j), [U1nRe(r), U1nIm(r), U2nRe(r), U2nIm(r)], 
numeric, maxmesh = 256) end do end do; F_r1 := F_r1, 
[Physics[`*`](k, r1), evalf(abs(F(Physics[`*`](Pi, 
Physics[`^`](2, -1)), Pi)))] end do; p2 := plot([F_r1], 
color = black, linestyle = dot); fk := proc (r) options 
operator, arrow; .75*((r-r0)^2/(r1-r0)^2+1) end proc; 
omega := .1; for i from 0 to 5 do for j from 0 to 5 do 
s[i, j] := dsolve(eval(eval({du16Im[1], du16Im[2], 
du16Re[1], du16Re[2], gru1Im[1], gru1Im[2], gru1Re[1], 
gru1Re[2], gru2Im[1], gru2Im[2], gru2Re[1], gru2Re[2]}, 
n = i), m = j), [U1nRe(r), U1nIm(r), U2nRe(r), U2nIm(r)], 
numeric, maxmesh = 256) end do end do; F_r1 := 
[Physics[`*`](k, r1), evalf(abs(F(Physics[`*`](Pi, 
Physics[`^`](2, -1)), Pi)))]; for omega by 50 to 10000 do 
for i from 0 to 5 do for j from 0 to 5 do s[i, j] := 
dsolve(eval(eval({du16Im[1], du16Im[2], du16Re[1], 
du16Re[2], gru1Im[1], gru1Im[2], gru1Re[1], gru1Re[2], 
gru2Im[1], gru2Im[2], gru2Re[1], gru2Re[2]}, n = i), 
m = j), [U1nRe(r), U1nIm(r), U2nRe(r), U2nIm(r)], numeric, 
maxmesh = 256) end do end do; F_r1 := F_r1, 
[Physics[`*`](k, r1), evalf(abs(F(Physics[`*`](Pi, 
Physics[`^`](2, -1)), Pi)))] end do; p3 := plot([F_r1], 
color = black, linestyle = dashdot):
\end{verbatim}

\end{enumerate}

\end{document}


\newpage
\setcounter{section}{0}
%\renewcommand{\thesection}{\arabic{section}.}
%Дифракция цилиндрических звуковых волн на упругой сфере с однородным покрытием.
\section{Теоретические сведения о дифракции}
\subsection{Постановка задачи}

\newpage
\subsection{Модель распространения волн}


\newpage
\subsection{Потенциал скоростей рассеянной волны}


\newpage

\section{Численное решение задачи о дифракции}



%
%
%\newpage
%
%\phantom{123}

\newpage


\section*{Заключение}
\addcontentsline{toc}{section}{Заключение}



\newpage

\section*{Приложение}
\addcontentsline{toc}{section}{Приложение}




\input{bibl.tex}

\end{document} 
